%!TEX root = template.tex

% configure the microtype font setting
\microtypesetup{
  final,        % always activate, even when in draft mode
  activate={true,nocompatibility},  % activate protrusion and expansion
  % tracking=true,  % This is disabled because it adds extra space around small caps. It's suppose to be disabled with \SetTracking{encoding={*}, shape=sc}{0}, but I can't verify this. Maybe you can figure it out!
  kerning=true,
  spacing=true,
  protrusion=true
  % more info on config to be found eg at http://www.khirevich.com/latex/microtype/
}
\microtypecontext{spacing=nonfrench}
% enable protrusion for superscript numbers
% (doesn't work with XeTeX right now)
% \SetProtrusion{
%   encoding={*},
%   family={cmr},
%   series={*},
%   size={6,7}
% }{
%   1={ ,750},2={ ,500},3={ ,500},4={ ,500},5={ ,500},
%   6={ ,500},7={ ,600},8={ ,500},9={ ,500},0={ ,500}
% }


% until where should titles be enumerated?
\setsecnumdepth{subsubsection}
\settocdepth{subsection}


% size parameters
\setlength{\textwidth}{150mm}       % width of the text area
\setlength{\textheight}{220mm}      % height of the text area
\setlength{\hoffset}{0pt}           % offset insertet at the left
\setlength{\voffset}{10pt}          % offset inserted at the top
\setlength{\oddsidemargin}{9mm}     % offset inserted to the left for odd pages
\setlength{\evensidemargin}{0mm}    % offset inserted to the left for even pages
\setlength{\marginparwidth}{18mm}   % width for marginal notes (Randnotizen)
\setlength{\parskip}{\bigskipamount} % space between two paragraphs
\setlength{\parindent}{10pt}        % indention for a new paragraph


% colors
\definecolor{gray75}{gray}{0.75}
\definecolor{gray90}{gray}{0.90}
\definecolor{darkblue}{rgb}{0,0,0.5}
\definecolor{darkgreen}{rgb}{0,0.5,0}
\definecolor{green}{rgb}{0.2,0.8,0.2}
\definecolor{red}{rgb}{1.0,0.3,0.0}
\definecolor{orange}{rgb}{1.0,0.6,0.0}


% some PDF settings (like hyperlinks in color)
\hypersetup{
 % pdftex,                   % use pdftex backend
 pdfencoding=auto,         % encoding
 pdftitle={\Pdftitle},     % title
 pdfauthor={\Author},      % author
 bookmarksnumbered=true,   % put section numbers in bookmarks
 % linktocpage=true,         % make page number (not text) be link on TOC
 % pdfpagelabels=true,       % set PDF page labels
 colorlinks=true,          % false: boxed links; true: colored links
 linkcolor=darkblue,       % color of internal links
 citecolor=darkblue,       % color of links to bibliography
 filecolor=darkblue,       % color of file links
 urlcolor=darkblue,        % color of external links
 draft=false               % always show links, even in draft mode
}

% settings for siunitx, see http://www.ctan.org/pkg/siunitx
\sisetup{
  per-mode = symbol,       % creates a backslash for division as default, can be overridden individually
  separate-uncertainty = true,  % dont use "10.0(1)" notation for uncertainties, but do use "10.0 \pm 0.1"
  binary-units = true,      % load also binary units like bit/byte
  list-final-separator = {, and~}  % use Oxford comma for \SIlist-s
}


% chapter definition
\newif\ifchapternonum
\makechapterstyle{jenor}{
  \renewcommand\printchaptername{}
  \renewcommand\printchapternum{}
  \renewcommand\printchapternonum{\chapternonumtrue}
  \renewcommand\chaptitlefont{\fontseries{bx}%
    \fontshape{n}\fontsize{25}{35}\selectfont\raggedleft}
  \renewcommand\chapnumfont{\fontseries{m}\fontshape{n}%
    \fontsize{3cm}{0in}\selectfont\color{gray75}}
  \renewcommand\printchaptertitle[1]{%
    \noindent%
    \ifchapternonum%
    \begin{tabularx}{\textwidth}{X}%
      {\parbox[b]{\linewidth}{\chaptitlefont ##1}%
      \vphantom{\raisebox{-15pt}{\chapnumfont 1}}}
    \end{tabularx}%
    \else
    \begin{tabularx}{\textwidth}{Xl}
      {\parbox[b]{\linewidth}{\chaptitlefont ##1}}
      & \raisebox{-15pt}{\chapnumfont\ \thechapter}%
    \end{tabularx}%
    \fi
    \par\vskip2mm\hrule
  }
}
\chapterstyle{jenor}


% other title definitions
% change the following to switch to sans serif title styles
% \setsecheadstyle{\fontfamily{\sfdefault}\Large\bfseries}
% \setsubsecheadstyle{\fontfamily{\sfdefault}\large\bfseries}
% \setsubsubsecheadstyle{\fontfamily{\sfdefault}\bfseries}
% \setparaheadstyle{\fontfamily{\sfdefault}\bfseries}
% \setsubparaheadstyle{\fontfamily{\sfdefault}\bfseries}


% table of contents
% \renewcommand{\cftsectionfont}{\fontfamily{\sfdefault}\selectfont\small}
% \renewcommand{\cftsubsectionfont}{\fontfamily{\sfdefault}\selectfont\small}


% caption styling
\captiontitlefont{\small}
\captionnamefont{\small\bfseries}
\indentcaption{10pt}

% Subfloats for figures (memoir option)
\newsubfloat{figure}

% rotate labels from showlabels to make them fit on the margin
% \renewcommand{\showlabelfont}{\scriptsize\ttfamily}
% \renewcommand{\showlabelsetlabel}[1]{\begin{turn}{60}\showlabelfont #1\end{turn}}


% Bibliography options
% \bibliographystyle{custom}

% Glossaries options
\renewcommand*{\glspostdescription}{}  % redefine the character displayed after an glossary entry from dot to nothing
\makeglossaries  % ensure glossary files are created
\glstoctrue  % Put glossaries into TOC
% to change the title of your acronyms uncomment the following lines and the correspoinding \makeglossaries in the main document
 %\providecommand{\myacronymtitle}{List of Acronyms}

% Makes the header not italics
\makeevenhead{headings}{\thepage}{}{\leftmark}
\makeoddhead{headings}{\rightmark}{}{\thepage}

% Customize todonotes
% First, get rid of the rounded corners
\makeatletter
\tikzstyle{notestyleraw} = [
draw=\@todonotes@currentbordercolor,
fill=\@todonotes@currentbackgroundcolor,
line width=0.5pt,
text width = \@todonotes@textwidth - 1.6 ex - 1pt,
inner sep = 0.8 ex,
rounded corners=0pt]
\makeatother
% Second, change default colors
\presetkeys{todonotes}{backgroundcolor=brightorange, bordercolor=orange,
prepend, caption={\textbf{\textcolor{darkorange}{NOTE}}}}{}



