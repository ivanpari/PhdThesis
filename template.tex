% ======================================================= %
%          Phd Thesis of Isis Van Parijs             %
%                                                         %
%  Structure:                                             %
%   template.tex          (main file)                     %
%    - _packages.tex      (load packages)                 %
%    - _settings.tex      (basic configuration)           %
%    - _newcommands.tex   (custom commands)               %
%    - 1 - Introduction/  (chapters in subdirectories)    %
%    - 2 - ...                                            %
%    -                                                    %
%                                                         %
% ======================================================= %

% draft or final?
\newcommand{\status}{draft} % balck lines next to overfull boxes
% If the draft option is passed to the package, all micro-typographic extensions will be disabled, which may lead to different line, and hence page, breaks. The draft and final options may also be inherited from the class options; of course, you can override them in the package options. E. g., if you are using the class option draft to show any overfull boxes, you should load microtype with the final option.

% The most general information for this document
\newcommand{\Title}{A search for flavour changing neutral currents involving a top quark and a Z boson, using the data collected by the CMS collaboration at a centre of mass of 13 TeV}
\newcommand{\Pdftitle}{\Title}  % For if you want to have a different title in the pdf document (e.g. special characters removed)
\newcommand{\Author}{Van Parijs, Isis}
\newcommand{\keywords}{LaTeX, template, CMS, FCNC, phd, thesis}
\newcommand{\refereeOne}{Prof.~Dr.~J. D'Hondt}
%\newcommand{\refereeTwo}{Prof.~Dr.~P. Thagoras}
\newcommand{\dateHandIn}{10~November~2017}
\newcommand{\dateDefense}{10~December~2017}


% load the general configuration
\input{_packages}
\input{_settings}
%!TEX root = template.tex

% fancy header



% general commands
\renewcommand{\>}{\frqq}
\newcommand{\<}{\flqq}
\newcommand{\myquote}[1]{\frqq{}#1\flqq{}\xspace}
\newcommand{\mysinglequote}[1]{\frq#1\flq{}\xspace}

% shortcuts for references
\newcommand{\fig}[1]{\hyperref[#1]{Figure \ref*{#1}}}
\newcommand{\tab}[1]{\hyperref[#1]{Table \ref*{#1}}}
\newcommand{\eq}[1]{\hyperref[#1]{Equation \ref*{#1}}}
\newcommand{\Sec}[1]{\hyperref[#1]{Section \ref*{#1}}}

% circled numbers
\newcommand*\circled[1]{\tikz[baseline=(char.base)]{
            \node[shape=circle,draw,inner sep=1.5pt] (char) {#1};}}

% citation needed
\newcommand{\todocite}[0]{\todo[color=green!40]{Add source}}


%abbreviations
\newcommand{\SM}{SM}
\newcommand{\BSM}{BSM}
\newcommand{\pu}{pileup}
% useful commands for equations
\newcommand{\e}{\enspace}  % width of one digit
\newcommand{\dd}{\,\mathrm{d}}
% \newcommand{\del}{\partial}  % comes from package commath
\newcommand{\Ra}{\Rightarrow}
\newcommand{\ra}{\rightarrow}

%\renewcommand{\vec}[1]{\underline{#1}}
\newcommand{\heq}{\widehat{=}}
\renewcommand{\deg}{$\,^\circ$}
\newcommand{\chindof}{\chi^2/\mathrm{ndof}}
\newcommand{\com}{$\sqrt{s}$ }
\newcommand{\impuls}{$\vec{p}$ }
\newcommand{\trimpuls}{\ensuremath{\vec{p}_{\mathrm{T}}} }
\newcommand{\psrap}{$\eta$ }
\newcommand{\abspsrap}{$\left| \eta\right|$ }

\newcommand{\xcal}[1]{\text{\usefont{OMS}{cmsy}{m}{n}#1}}  % Use original mathcal font instead of charters mathcal version
\newcommand{\orderof}[1]{$\xcal{O}\left( #1 \right)$}  % \orderof{\si{\metre}} now is more beautiful
\newcommand{\BR}{\ensuremath{\mathcal{B}}\xspace}
\newcommand{\FCNC}{FCNC}
% new units for siunitx package
\newcommand{\qe}{\ensuremath{q_{\mathrm{e}}}\xspace}

\DeclareSIUnit\event{ev}
\DeclareSIUnit\channel{ch}
\DeclareSIUnit\MeV{MeV}
\DeclareSIUnit\GeV{GeV}
\DeclareSIUnit\TeV{TeV}
\DeclareSIUnit\Tesla{T}
\DeclareSIUnit{\atm}{atm}
\DeclareSIUnit{\ADC}{ADC}
\DeclareSIUnit{\pb}{pb}
% particles
\newcommand{\muR}{\ensuremath{\mu_{\mathrm{R}}}}
\newcommand{\muF}{\ensuremath{\mu_{\mathrm{F}}}}
\newcommand{\st}{single top}
\renewcommand{\tt}{top pair }
\newcommand{\Lagr}{\ensuremath{\mathcal{L}}}
\newcommand{\lagr}{\ensuremath{\mathcal{L}}}
\newcommand{\like}{\ensuremath{\mathfrak{L}}}
\newcommand{\lumi}{\ensuremath{\mathrm{L}}}
\newcommand{\SSU}{\ensuremath{\mathrm{SU}_{\mathrm{C}}(3) \times \mathrm{SU}_{\mathrm{L}}(2) \times \mathrm{U}_{\mathrm{Y}}(1)}}
\newcommand{\Uone}{\ensuremath{\mathrm{U}_{\mathrm{Y}}(1)}}
\newcommand{\SU}{\ensuremath{\mathrm{SU}_{\mathrm{L}}(2) \times \mathrm{U}_{\mathrm{Y}}(1)}}
\newcommand{\Sthree}{\ensuremath{\mathrm{SU}_{\mathrm{C}}(3)}}
\newcommand{\Stwo}{\ensuremath{ \mathrm{SU}_{\mathrm{L}}(2)}}
\newcommand{\Bfield}{\ensuremath{ \mathrm{B}_{\mu}}}
\newcommand{\Wfieldone}{\ensuremath{ \mathrm{W}_{\mu}^1}}
\newcommand{\Wfieldtwo}{\ensuremath{ \mathrm{W}_{\mu}^2}}
\newcommand{\Wfieldthree}{\ensuremath{ \mathrm{W}_{\mu}^3}}
\newcommand{\Gfields}{\ensuremath{ \mathrm{G}_{\mu}^{1...8}}}
\newcommand{\Gtensord}{\ensuremath{ \mathrm{G}_{\mu\nu}^{i}}}
\newcommand{\Gtensor}{\ensuremath{ \mathrm{G}_{\mu\nu}}}
\newcommand{\Ztensor}{\ensuremath{ \mathrm{Z}_{\mu\nu}}}
\newcommand{\Gtensoru}{\ensuremath{ \mathrm{G}^{\mu\nu i}}}
\newcommand{\Wtensord}{\ensuremath{ \mathrm{W}_{\mu\nu}^{i}}}
\newcommand{\Wtensoru}{\ensuremath{ \mathrm{W}^{\mu\nu i}}}
\newcommand{\Btensord}{\ensuremath{ \mathrm{B}_{\mu\nu}^{i}}}
\newcommand{\Btensoru}{\ensuremath{ \mathrm{B}^{\mu\nu i}}}
\newcommand{\photonfield}{\ensuremath{ \mathrm{A}_{\mu}}}
\newcommand{\photontensor}{\ensuremath{ \mathrm{A}_{\mu\nu}}}
\newcommand{\Zfield}{\ensuremath{ \mathrm{Z}^0_{\mu}}}
\newcommand{\Wfield}{\ensuremath{ \mathrm{W}_{\mu}^{\pm}}}
\newcommand{\sW}{\ensuremath{\mathrm{sin } \theta_{\mathrm{W}}}}
\newcommand{\cW}{\ensuremath{\mathrm{cos } \theta_{\mathrm{W}}}}
\newcommand{\mtw}{\ensuremath{m_{\mathrm{T}}(\W)}\xspace}
\newcommand{\mZ}{\ensuremath{m_{\mathrm{Z}}}\xspace}
\newcommand{\LSM}{\ensuremath{\Lagr_\mathrm{SM}}}
\DeclareRobustCommand{\order}{\ensuremath{\mathcal{O}}}
\newcommand{\tZq}{\ensuremath{\mathrm{tZ}\Pquark}}
\newcommand{\kgqt}{\ensuremath{\kappa_{\Pgluon\Pquark\Ptop}}}
\newcommand{\kZqt}{\ensuremath{\kappa_{\Ptop\PZ\Pquark}}}
\newcommand{\eHqt}{\ensuremath{\eta_{\Ptop\PHiggs\Pquark}}}
\newcommand{\kfqt}{\ensuremath{\kappa_{\Ptop\Pphoton\Pquark}}}
\newcommand{\kgqtl}{\ensuremath{\kappa_{\Pgluon\Pquark\Ptop}/\Lambda}}
\newcommand{\kZqtl}{\ensuremath{\kappa_{\Ptop\PZ\Pquark}/\Lambda}}
\newcommand{\kfqtl}{\ensuremath{\kappa_{\Ptop\Pphoton\Pquark}/\Lambda}}
\newcommand{\zZqt}{\ensuremath{\zeta_{\mathrm{\Ptop\PZ\Pquark}}}\xspace}
\newcommand{\zZut}{\ensuremath{\zeta_{\mathrm{\Ptop\PZ\Pup}}}\xspace}
\newcommand{\zZct}{\ensuremath{\zeta_{\mathrm{\Ptop\PZ\Pcharm}}}\xspace} 
\newcommand{\zWqt}{\ensuremath{\zeta_{\mathrm{\Ptop\PW\Pquark}}}\xspace} 
\newcommand{\kxqt}{\ensuremath{\kappa_{\Ptop \mathrm{X} \Pquark}}/\Lambda\xspace}
\newcommand*\textfrac[2]{              % display fraction in textstyle
	\frac{\text{#1}}{\text{#2}}
}

\newenvironment{chapquote}[2][2em]
{\setlength{\@tempdima}{#1}%
	\def\chapquote@author{#2}%
	\parshape 1 \@tempdima \dimexpr\textwidth-2\@tempdima\relax%
	\itshape}
{\par\normalfont\hfill--\ \chapquote@author\hspace*{\@tempdima}\par\bigskip}


% particles
\renewcommand{\Pquark}{\ensuremath{\mathrm{q}}}
\newcommand{\ttbar}{\ensuremath{\mathrm{t}\bar{\mathrm{t}}}}
\newcommand{\qqbar}{\ensuremath{\mathrm{q}\bar{\mathrm{q}}}}
\newcommand{\tZ}{\ensuremath{\mathrm{tZ}}}
\newcommand{\tbarZ}{\ensuremath{\bar{\mathrm{t}}\mathrm{Z}}}
\newcommand{\bbbar}{\ensuremath{\mathrm{b}\bar{\mathrm{b}}}}
\newcommand{\ttZ}{\ensuremath{\ttbar\mathrm{Z}}}
\newcommand{\ttW}{\ensuremath{\ttbar\mathrm{W}}}
\newcommand{\ttH}{\ensuremath{\ttbar\mathrm{H}}}
\newcommand{\WZ}{\ensuremath{\mathrm{WZ}}}
\newcommand{\WZZ}{\ensuremath{\mathrm{WZZ}}}
\newcommand{\WW}{\ensuremath{\mathrm{WW}}}
\newcommand{\WWZ}{\ensuremath{\mathrm{WWZ}}}
\newcommand{\ZZ}{\ensuremath{\mathrm{ZZ}}}
\newcommand{\ZZZ}{\ensuremath{\mathrm{ZZZ}}}
\newcommand{\tWZ}{\ensuremath{\mathrm{tWZ}}}
\newcommand{\tqH}{\ensuremath{\mathrm{tqH}}}
\newcommand{\DY}{\ensuremath{\PZ/\Pgammastar+\mathrm{jets}}}
% Software
\newcommand{\Geant}{\texttt{Geant}}
\newcommand{\aMCMG}{\texttt{MadGraph5$\_$aMC$@$NLO}}
\newcommand{\MG}{\texttt{MadGraph}}
\newcommand{\aMC}{\texttt{aMC$@$NLO}}
\newcommand{\Powheg}{\texttt{POWHEG}}
\newcommand{\MS}{\texttt{MadSpin}}
\newcommand{\Pythia}{\texttt{Pythia}}
\newcommand{\FR}{\texttt{FeynRules}}
\newcommand{\UFO}{\texttt{Universal Feynrules Output}}
\newcommand{\CTEQ}{\texttt{CTEQ}}
\newcommand{\JHU}{\texttt{JHU}}
\newcommand{\ME}{\texttt{MadEvent}}

%\addbibresource{references.bib}  % Use with BibLaTeX
\addbibresource{Bibliography/Introduction.bib}
\addbibresource{Bibliography/Chapter2.bib}
\addbibresource{Bibliography/CMS.bib}
\addbibresource{Bibliography/analysistechniques.bib}


%!TEX root = template.tex

%\newglossaryentry{gls:template}{
%    name = {this template},
%    description = {This awesomely great template thing}
%}
%\newacronym{jama}{JAMA}{Just A Meaningless Acronym}
%\newacronym[prefixfirst={the~}]{aalf}{AALF}{Arms And Legs Facility}

%\newacronym{hcal}{HCAL}{hadronic calorimeter}
\newglossaryentry{gls:hcal}{
    name = {HCAL},
    description = {Hadroni calorimeter}
}
% -------------------------------------------------- %
%    title, TOC, ...                                 %
% -------------------------------------------------- %

% begin the document
\begin{document}
\frontmatter



% the title page
%!TEX root = template.tex

% \begin{titlepage}
\begin{center}
	\vspace*{5mm}
     
    \begin{figure}[ht]
    	\centering
    	\includegraphics[width=0.5\linewidth]{"VUB MONO positief/VUB MONO POSITIEF OUTLINE"}
    %	\caption{}
    	\label{fig:vub-mono-positief-outline}
    \end{figure}
    
	\huge \textbf{\Title}

	\vspace{10mm}

	\Large \Author
	
	\vspace{15mm}
	\large \textbf{Proefschrift ingediend met het oog op het behalen van de academische graad Doctor in de Wetenschappen.}

	\vspace{25mm}
	\small
	\begin{tabular}{rl}
     Published in & Faculteit Wetenschappen \& Bio-ingenieurswetenschappen \\[2mm]
                  &\large Vrije Universiteit Brussel \\[2mm]
%                  & Departement ELEM \\ [2mm]
 %           Issue & \large 18e9 Rev. 7 \emph{Cataclysm} \\[2mm]
               At & \large 10 November 2017.\\[15mm]
   Responsible Contact: & \large Isis Van Parijs \\[1mm]
                  & Interuniversitary Intitute for High Energy Physics\\
                  Supervisor: & Prof. Jorgen D'Hondt
	\end{tabular}


\end{center}
% \end{titlepage}

\thispagestyle{empty}
\newpage
\null

Doctoral examination commission:\\

\begin{tabular}{l l @{\hspace{1cm}} l l}
	Chair:& & Prof. Dr. B. Craps & Vrije Universiteit Brussel \\
	Supervisor: && Prof. Dr. J. D'Hondt & Vrije Universiteit Brussel \\ 
	Secretary:& & Prof. Dr. S. Buitink & Vrije Universiteit Brussel \\
	Other:& & & \\
	&DNTK: & Prof. Dr. P. Van Mulders & Vrije Universiteit Brussel  \\
	&DSCH: & Prof. Dr. S. Ballet & Vrije Universiteit Brussel  \\
	&External & Prof. Dr. B. Fuks  & University Pierre and Marie Curie (Paris VI) \\
	&External: & Prof. Dr. A. Onofre & LIP and University do Minho 
\end{tabular}


\vfill
\begin{tabular}{l @{\hspace{1cm}} l l}
	Date of Hand-in: & \dateHandIn &\\
	Date of Private Defense: & \dateDefense &
\end{tabular}
\vspace{10mm}

\textcopyright\ 2017 Isis Van Parijs\\
All rights reserved. No parts of this book may be reproduced or transmitted in any form or by any means, electronic, mechanical, photocopying, recording, or otherwise, without the prior written permission of the author.
% \newpage
% \thispagestyle{empty}
% \mbox{}
\cleardoublepage{}

\setlength{\topmargin}{0mm}
\normalsize%\rm


% table of contents
%Protrusion (margin kerning, hanging punctuation) enables characters to cross the margin edge in order to increase uniformity of the optical appearance of text boundaries (see pages 39–50 of Hàn Thế Thành’s dissertation). After activating protrusion with the option activate={true,nocompatibility} of microtype, crossing the margins is best seen for the characters having small area (".", ",", "-") or highly non-flat contour near the margin edge ("y", "r" near the right margin). This is illustrated in the figure below where comma and "r" cross the right margin edge indicated by the red dashed line:
\microtypesetup{protrusion=false} % disables protrusion locally in the document
\tableofcontents*
\microtypesetup{protrusion=true}  % enables protrusion again
\newpage

% load glossaries


% -------------------------------------------------- %
%    the main content                                %
% -------------------------------------------------- %

\mainmatter
\chapter[An introduction to the theory]{Theoretical basis}
% top quark 
% http://www.slac.stanford.edu/pubs/slacreports/reports03/ssi95-001.pdf
%https://d22izw7byeupn1.cloudfront.net/files/RevModPhys.69.137.pdf
%Disentangling Heavy Flavor at Colliders
% https://arxiv.org/abs/1702.02947

%Why dim 6 operators? 
%https://indico.cern.ch/event/610458/#4-top-16-017-eft-re-interpreta
% particle physics
%http://www.hep.phy.cam.ac.uk/~thomson/lectures/partIIIparticles/Handout13_2009.pdf
%EFT Theory
%https://indico.cern.ch/event/537012/timetable/#day-2016-11-23
%https://indico.cern.ch/event/612805/

%eft interpretation 
% https://mail.google.com/mail/u/0/?tab=wm#inbox/15b57592ac482483
% https://twiki.cern.ch/twiki/bin/view/CMS/TopEFT
The Standard Model (\SM)~\cite{Peskin:257493} 
is a name given in 1970s to a theory describing the fundamental particles and their interactions. This quantum field theory describes the particles and their interactions as fields and has successfully incorporated three of the four fundamental forces in the universe. In \Sec{sec:SMcontent}, the particle content of the \SM\ is summarised, while \Sec{sec:SMlagr} describes  the \SM\ Lagrangian and its symmetries. In \Sec{sec:FCNC}, the flavour content of the \SM\ is highlighted. The successful theory of the \SM\ has some shortcomings which are discussed in \Sec{sec:BSM} and lead to searches for a more general theory. One of such a search is using effective field theory (EFT). In \Sec{sec:EFT} an EFT model focussing on flavour changing neutral currents (FCNC) involving a top quark is presented. Its current experimental constraints are given in \Sec{sec:ExpConstr}.


\section{Elementary particles and forces}
\label{sec:SMcontent}
The interactions in nature can be described by four forces, the strong force, the electromagnetic (EM) force, the weak force and the gravitational force, that are mediated by particles with an integer spin, bosons. The strong interaction is mediated by eight gluons \Pgluon, while the electromagnetic force is mediated by photons \Pphoton, and the weak force by \PZ and \PWpm bosons. In \tab{tab:forces}, the forces and their characteristics are shown. The gravitational force is the only force not included in the \SM\ and can be neglected for energies lower than the Planck scale (1.22 $10^{19}$ \GeV).
\begin{table}[htbp]
	\centering
	\caption{The four forces of nature and their characteristics.}
	\begin{tabular}{lcc}
		\toprule
		& Range & Mediator \\ 
		\midrule
		Strong force & $10^e{-15}$ \m & 8 gluons  \\ 
	
		Electromagnetic force & $\infty$ & photon  \\ 
		 
		Weak force & $10^{-18}$ \m & \PWpm, \PZ bosons \\ 
		
		Gravitational force & $\infty$ & unknown \\ 
		\bottomrule
	\end{tabular} 
	\label{tab:forces}
\end{table}

The fermions are the particles that make up the visible matter in the universe. They carry half integer spin and can be subdivided into leptons and quarks, where leptons don't interact strongly. Each fermion has a corresponding anti-fermion which has the same mass and is oppositely charged. The electron \Pelectron is the first elementary particle discovered~\cite{electrondiscovery} and belongs to the first generation of leptons together with electron neutrino \Pnue. The second generation is made up of the muon \Pmuon and the muon neutrino \Pnum, whereas the third generation consists of the tau \Ptau and the tau neutrino \Pnut. The neutrino's are neutral particles, while the other leptons have charge $\pm \qe$ where \qe represents the elementary charge of 1.602 $10^{-19}$ C. The masses of the charged leptons differ by four orders of magnitude between the first and third generations.In the \SM\ the neutrino's are assumed to be massless, while it is experimentally established that neutrino do have a tiny non-zero mass. In \tab{tab:leptongen}, the leptons and their properties in the \SM\ are summarised. 
\begin{table}[htbp]
	\centering
	\caption{The properties of the leptons in the three generations of the \SM~\cite{PDG}, where \qe represents the elementary  charge.}
	\begin{tabular}{lccc}
		\toprule
		Generation & Particle  & Mass  & Charge \\ 
		\midrule
		\multirow{2}{*}{First} & \Pelectron & 0.511 \MeV & -\qe  \\ 
		& \Pnue & $\approx$ 0 & 0\\
		
	\multirow{2}{*}{Second} & \Pmuon & 106 \MeV &-\qe  \\ 
	& \Pnum & $\approx$ 0 & 0\\
	
	\multirow{2}{*}{Third} & \Ptau & 1 777 \MeV & -\qe  \\ 
	& \Pnut & $\approx$ 0 & 0 \\
	
		
		\bottomrule
	\end{tabular} 
	\label{tab:leptongen}
\end{table}

The quarks can also be divided into three generations. Unlike the leptons, they carry colour charge and can interact via the strong interaction. The top quark, discovered in 1995 at the Tevatron~\cite{observationtopD0,observationtopCDF} is the heaviest \SM\ particle with a mass close to $173.1\pm0.6$ \GeV\footnote{In this thesis all masses and energies are expressed in natural units, where the speed of light and $\hbar$ are taken to be equal to one.}~\cite{PDG}. The quarks and their properties are summarized in \tab{tab:quarkgen}. 
\begin{table}[htbp]
	\centering
	\caption{The properties of the quarks in the three generations of the \SM~\cite{PDG}, where \qe represents the elementary  charge.}
	\begin{tabular}{lccc}
		\toprule
		Generation & Particle  & Mass  & Charge \\ 
		\midrule
		\multirow{2}{*}{First} & up \Pup &$2.2_{-0.4}^{+0.6}$ \MeV& $\textfrac{2}{3}$ \qe  \\ 
		& down \Pdown & $4.7^{+0.5}_{-0.4}$ \MeV & $\textfrac{-1}{3}$ \qe\\
		
		\multirow{2}{*}{Second} & charm \Pcharm & 1.28 $\pm$ 0.03 \GeV &$\textfrac{2}{3}$ \qe  \\ 
		& strange \Pstrange & $96^{+8}_{-4}$ \MeV & $\textfrac{-1}{3}$ \qe\\
		
		\multirow{2}{*}{Third} & top \Ptop & 173.1 $\pm$ 0.6 \GeV &$\textfrac{2}{3}$ \qe  \\ 
		&bottom \Pbottom & $4.18^{+0.04}_{-0.03}$ \GeV & $\textfrac{-1}{3}$ \qe \\
		
		
		\bottomrule
	\end{tabular} 
	\label{tab:quarkgen}
\end{table}

The scalar boson, commonly known as the Higgs boson, is the last piece of the \SM\ and is discovered in 2012~\cite{Chatrchyan:2012xdj,Aad:2012tfa}. It is responsible for the masses of the \PWpm and \PZ boson, and that of the fermions.


\section{Standard Model Lagrangian}
\label{sec:SMlagr}
The \SM\ is a quantum field theory and thus describes the dynamics and kinematics of particles and forces by a Lagrangian \Lagr. The theory is based on the \SSU\ gauge symmetry, where \SU\ describes the electroweak interaction and \Sthree\ the strong coupling. The indices refer to colour C, the left chiral nature of the \Stwo\ coupling L, and the weak hypercharge Y. Its Lagrangian is constructed such that contains symmetries representing physics conservation laws such as conservation of energy, momentum and angular momentum. By imposing gauge invariance {\todo{should I explain gauge invariance or is a reference enough?}} the symmetries under local group transformations are sustained. 



The \Uone\ group has one generator Y with an associated gauge field \Bfield. The three gauge fields \Wfieldone, \Wfieldtwo, and \Wfieldthree, are associated to \Stwo with three generators that can can  be written as half of the Pauli matrices: 
\begin{equation}
T_1 =  \frac{1}{2}
\begin{pmatrix}
0  &  1      \\
1  & 0      
\end{pmatrix}, \;
T_2= \frac{1}{2}
\begin{pmatrix}
0  &  -i     \\
i  &  0      
\end{pmatrix},\;\mathrm{ and } \;
 T_3= \frac{1}{2}
 \begin{pmatrix}
 1  &  0     \\
 0  &  -1 
 \end{pmatrix}.
\end{equation}
The generators $T^a$ satisfy the Lie algebra: 
\begin{equation}
 \left[T^a,T^b\right] = i \epsilon^{abc} T_c \; \mathrm{ and } \left[T^a, Y\right] = 0, 
\end{equation}
where $\epsilon^{abc}$ is an antisymmetric tensor. The gauge fields of \Stwo\ only couple to left-handed fermions as required by the observed parity violating nature of the weak force. The \Sthree\ group represents quantum chromodynamics (QCD). It  has eight generators corresponding to eight gluon fields \Gfields. Unlike \SU, \Sthree\ is not chiral. 

Under \Sthree\, quarks are colour triplets while leptons are colour singlets. This implies that the quarks carry a colour index ranging between one and three, whereas leptons do not take part in strong interactions. Based on the chirality, the quarks and leptons are organized in doublets or singlets. Each generation $i$ of fermions consists of these left-handed doublets and right-handed singlets: 
\begin{equation}
\mathrm{l}_{\mathrm{L}} =  
\begin{pmatrix}
\Pelectron_{\mathrm{L}}       \\
\Pneutrino_{\mathrm{L}}     
\end{pmatrix}, \; \Pelectron_{\mathrm{R}}, \; \mathrm{q}_{\mathrm{L}} = 
\begin{pmatrix}
\Pup_{\mathrm{L}}       \\
\Pdown_{\mathrm{L}}     
\end{pmatrix}, \; \Pup_{\mathrm{R}}, \; \mathrm{and} \; \Pdown_{\mathrm{R}}
\end{equation}

The \SM\ Lagrangian can be decomposed as a sum of four terms
\begin{equation}
\lagr_{\mathrm{SM}} = \lagr_{\mathrm{gauge}} + \lagr_{\mathrm{f}} + \lagr_{\mathrm{Yuk}} + \lagr_{\phi}, 
\end{equation}
that are related to the gauge, fermion, Yukawa and scalar sectors. The gauge Lagrangian regroups the gauge fields of all three symmetry groups, and the fermionic part consists of kinetic energy terms for quarks and leptons. The interaction between fermions and the scalar doublet $\phi$ gives rise to fermion masses and is described by the Yukawa Lagrangian. The scalar part of the Lagrangian is composed of a kinematic and potential component related to the scalar boson. 
%\begin{equation}
% \lagr_{\mathrm{gauge}} = -\frac{-1}{4} \Gtensord \Gtensoru -\frac{-1}{4} \Wtensord \Wtensoru - -\frac{-1}{4} \Btensord \Btensoru, 
%\end{equation}
%where the tensors are
%\begin{align}
%\Gtensord &= \partial_{\mu}\mathrm{G}_{\nu}^i - \partial_{\nu}\mathrm{G}_{\mu}^i - g_{\mathrm{s}} f_{ijk} \mathrm{G}_{\mu}^j \mathrm{G}_{\nu}^k, \; \mathrm{ with }\; i,j,k = 1,...,8 \\
%\Wtensord &= \partial_{\mu}\mathrm{W}_{\nu}^i - \partial_{\nu}\mathrm{W}_{\mu}^i - g_{\mathrm{s}} \epsilon_{ijk} \mathrm{W}_{\mu}^j \mathrm{G}_{\nu}^k, \; \mathrm{ with }\; i,j,k = 1,...,8 \\
%\end{align}

For the electroweak theory, two coupling constants are introduced, namely $g'$ for \Uone\ and $g$ for \Stwo. The physically observable gauge bosons of this theory are the photonfield \photonfield, the \Zfield, and \Wfield. These are a superposition of the four gauge fields of \SU: 
\begin{equation}
\photonfield = \sW \Wfieldone + \cW \Bfield, \; \Zfield = \cW \Wfieldthree - \sW \Bfield, \; \mathrm{ and } \; \Wfield = \sqrt{\frac{1}{2}}\left(\Wfieldone\mp \Wfieldtwo\right), 
\end{equation}
where $\theta_{\mathrm{W}}$ represents the weak mixing angle defined as $\mathrm{tan} \theta_{\mathrm{W}} = \frac{g'}{g}$.

The coupling constant representing the strength of the QCD interactions is denoted as $g_{\mathrm{s}}$. In QCD their is asymptotic freedom whereby the strong coupling constant becomes weaker as the energy with which the interaction between strongly interacting particles is probed increases, and stronger as the distance between the particles increases. A consequence of this is known as colour confinement. The quarks and gluons can not exist on their own and are not observed individually. They are bound in colour neutral states called hadrons, this process is known as hadronisation. 
\subsection*{Electroweak symmetry breaking}
In $\lagr_{\mathrm{gauge}}$ and $\lagr_{\mathrm{f}}$ are no mass terms for fermions present because only singlets under \SSU\ can acquire a mass with an interaction of the type $m^2\phi^{\dagger}\phi$ without breaking the gauge invariance. In order to accommodate mass terms for fermions and gauge fields, electroweak symmetry breaking, leading to $\lagr_{\phi}$ is introduced. 

The scalar doublet is introduced in the \SM\ as 
\begin{equation}
\phi = \frac{1}{\sqrt{2}}
\begin{pmatrix}
\varphi_1 + i \varphi_2    \\
\varphi_3 + i \varphi_4    
\end{pmatrix}.
\end{equation}
Its field potential is of the form \todo{check if I need to add constants here}
\begin{equation}
V(\phi) = \mu^2 \phi^{\dagger}\phi + \lambda(\phi^{\dagger}\phi)^2, 
\end{equation}
with $\mu^{2} <0$ and $\lambda$ a positive integer. This choice of parameters gives the potential a "Mexican hat" shape. I has an infinite set of minima (ground states) and by expanding the field around an arbitrary choice of ground state, the electroweak symmetry is broken (\cancel{EW}): 
\begin{equation}
\phi = 
\begin{pmatrix}
0    \\
\frac{v}{\sqrt{2}}    
\end{pmatrix}
+ \hat{\phi}, 
\end{equation}
where $v$ is the vacuum expectation value (vev), measured to be around 245 \GeV\ and corresponds to $\sqrt{\frac{-\mu}{\lambda}}$. The scalar doublet's four degrees of freedom is reduced to three degrees of freedom that couple to the gauge fields and mix with the \PWp, \PWm and \PZ bosons. The remaining fourth degree of freedom has given rise ta physically observable particle , called the Brout-Englert-Higgs (BEH) boson.
This spontaneous symmetry breaking leaves the gauge invariance intact and gives masses to the \PWpm and \PZ bosons as:
\begin{equation}
m_{\PW} = \frac{1}{2}v|g| \quad \mathrm{and} \quad m_{\PZ} = \frac{1}{2}v \sqrt{g'^2 + g^2}.
\end{equation}
The Brout-Englert-Higgs field couples universally fermions with a strength proportional to their masses, and to gauge bosons with a strength proportional to the square of their masses. 


\section{Flavour changing currents in the \SM}
\label{sec:FCNC}
%In the electroweak theory, the \PW boson is responsible for flavour changing charged currents, and the \PZ boson for 
%lading W -> smamak veranderen -> niet voor Z , integendeel verboden op tree level onderdrukt op higher order
Flavour changing charged currents are introduced in 1963 by Nicola Cabibbo \todocite. Via interaction with a \PW boson the flavour of the quarks is changed. At the time of the postulation, only up, down and strange quarks were known and the charged weak current was described as a coupling between the up quark and $\Pdown_{\mathrm{weak}}$, where $\Pdown_{\mathrm{weak}}$ is a linear combination of the down and strange quarks, $\Pdown_{\mathrm{weak}}= \mathrm{cos }\theta_{\mathrm{c}} \Pdown + \mathrm{sin }\theta_{\mathrm{c}} \Pstrange$. This linear combination is a direct consequence of the chosen rotation
\begin{equation}
\begin{pmatrix}
\Pdown_{\mathrm{weak}} \\
\Pstrange_{\mathrm{weak}} 
\end{pmatrix}
 = 
 \begin{pmatrix}
 \mathrm{cos }\theta_{\mathrm{c}} &  \mathrm{sin }\theta_{\mathrm{c}} \\
 - \mathrm{sin }\theta_{\mathrm{c}} &  \mathrm{cos }\theta_{\mathrm{c}}
 \end{pmatrix}
 \begin{pmatrix}
 \Pdown \\
 \Pstrange 
 \end{pmatrix} = \mathcal{R} 
 \begin{pmatrix}
 \Pdown \\
 \Pstrange 
 \end{pmatrix}, 
\end{equation}
where the rotation angle $\theta_{\mathrm{c}}$ is known as the Cabibbo angle. This provides a definition for the charged weak current between \Pup and \Pdown quarks, 
\begin{equation}
J_{\mu} = \bar{\Pmu} \gamma_{\mu}\left(1+\gamma_5\right)\Pdown_{\mathrm{weak}}. 
\end{equation} 
A consequence of Cabibbo's approach is that the $\Pstrange_{\mathrm{weak}}$ is left uncoupled, leading to Glashow, Iliopoulos and Maiani (GIM) \todocite to require the existence of a fourth quark with charge $\textfrac{2}{3}$. This quark, known as the charm quark, couples to $\Pstrange_{\mathrm{weak}}$ and a new definition of the charged weak current is modified to 
\begin{equation}
J_{\mu} = \begin{pmatrix}
u & c
\end{pmatrix}  \gamma_{\mu}\left(1+\gamma_5\right)\mathcal{R}  \begin{pmatrix}
d \\ s
\end{pmatrix}
= \bar{U} \gamma_{\mu}\left(1+\gamma_5\right)\mathcal{R}D. 
\end{equation} 

The neutral weak current is defined as 
\begin{equation}
J_{3} = \bar{U} \gamma_{\mu}\left(1+\gamma_5\right)\left[\mathcal{R}, \mathcal{R}^{\dagger}\right]D, 
\end{equation} 
and is diagonal in flavour space. This has as consequence that no flavour changing neutral currents occur at tree-level Feynmann diagrams\footnote{Feynmann diagrams are physical representation of interaction between particles. They are based on Feynmann rules~\cite{Peskin:257493}.} \todo{should I explain feynmann diagrams?}.


Kobayashi and Maskawa \todocite generalised the Cabibbo rotation matrix to accommodate for a third generation of quarks. The result is a $3\times 3$ unitary matrix known as the CKM matrix, responsible for the mixing of weak interaction states of down-type quarks: 
\begin{equation}
\begin{pmatrix}
\Pdown_{\mathrm{weak}} \\
\Pstrange_{\mathrm{weak}} \\
\Pbottom_{\mathrm{weak}}
\end{pmatrix}
= 
\begin{pmatrix}
V_{\Pup\Pdown} & V_{\Pup\Pstrange} & V_{\Pup\Pbottom} \\
V_{\Pcharm\Pdown} & V_{\Pcharm\Pstrange} & V_{\Pcharm\Pbottom} \\
V_{\Ptop\Pdown} & V_{\Ptop\Pstrange} & V_{\Ptop\Pbottom}
\end{pmatrix}
\begin{pmatrix}
\Pdown \\
\Pstrange \\
\Pbottom
\end{pmatrix} = \mathcal{V}_{\mathrm{CKM}} \begin{pmatrix}
\Pdown \\
\Pstrange \\
\Pbottom
\end{pmatrix}.
\end{equation}
The unitarity of the matrix ($\mathcal{V}_{\mathrm{CKM}}^{\dagger}\mathcal{V}_{\mathrm{CKM}} = \mathbb{1}$). A general $3\times 3$ unitary matrix depends on three real angles and six phases. For the CKM matrix, the freedom to redefine the phases of the quark eigenstates can remove fives of the phases, leaving a single physical phase known as the Kobayashi-Maskawa phase. This phase is responsible for the charge parity violation in the \SM~\cite{CKM}. 
% see wolfenstein parametrisation http://pdg.lbl.gov/2017/reviews/rpp2016-rev-cp-violation.pdf 13,53
Each element $V_{\mathrm{ij}}$ of $ \mathcal{V}_{\mathrm{CKM}}$ represents the transition probability of a quark i going to a quark j, and is experimentally determined to be~\cite{PDG}
\begin{equation}
\mathcal{V}_{\mathrm{CKM}} =
\begin{pmatrix}
0.97425 \pm 0.00022  & 0.2253 \pm 0.0008      & (4.13 \pm 0.49) 10^{-3} \\
0.225 \pm 0.008      & 0.986 \pm 0.016        & (41.1 \pm 1.3) 10^{-3} \\
(8.4\pm 0.6) 10^{-3} & (40.0 \pm 2.7) 10^{-3} & 1.021 \pm 0.032
\end{pmatrix}.
\label{eq:CKM}
\end{equation}

From  \eq{eq:CKM} follows that top quarks predominantly decay via charged weak currents to bottom quarks, with a probability consistently with unity. In the \SM, FCNC can only occur via higher loop Feynmann diagrams which are highly suppressed. The expected transition probabilities for a top quark decaying via a FCNC interaction in the \SM\ are given in \tab{tab:FCNCBR}, where it is clear that the FCNC sector of the \SM\ is still beyond the reach of the sensitivity of current experiments. 
\begin{table}[htbp]
	\centering
	\caption{The predicted branching ratios \BR\ for FCNC interactions involving the top quark in the \SM~\cite{AguilarSaavedra:2004wm}}
	\begin{tabular}{lclc}
		\toprule
	    Process	& \BR\ in the \SM  &  Process	& \BR\ in the \SM \\ 
		\midrule
		$ \Ptop \rightarrow \Pup \PZ $         & $8 \; 10^{-17}$  &	$ \Ptop \rightarrow \Pcharm \PZ $      & $1 \; 10^{-14}$   \\
		$ \Ptop \rightarrow \Pup \Pphoton $    & $4 \; 10^{-16}$  & $ \Ptop \rightarrow \Pcharm \Pphoton $ & $5 \; 10^{-14}$   \\
		$ \Ptop \rightarrow \Pup \Pgluon $     & $4 \; 10^{-14}$  & $ \Ptop \rightarrow \Pcharm \Pgluon $  & $5 \; 10^{-12}$  \\
		$ \Ptop \rightarrow \Pup \PHiggs $     & $2 \; 10^{-17}$  & $ \Ptop \rightarrow \Pcharm \PHiggs $  & $3 \; 10^{-15}$ \\
		\bottomrule
	\end{tabular} 
	\label{tab:FCNCBR}
\end{table}



\section{Motivations for new physics}
\label{sec:BSM}
\todo{Reread and elaborate}
Many high energy experiments confirm the success of the \SM. In particular the scalar boson, the cornerstone of the \SM, has consecrated the theory. Unfortunately there are also strong indications that the \SM\ ought to be a lower energy expression of a more global theory. The existence of physics beyond the \SM (BSM)~\cite{BSMWiley} is strongly motivated. These motivations are based on direct evidence from observation such as the existence of neutrino masses, the existence of dark matter and dark energy, or the matter-antimatter asymmetry, and also from theoretical problems such as the hierarchy problem, the coupling unification or the large numbers of free parameters in the \SM. 


In the \SM, the neutrino is assumed to be massless, whilst experiments with solar, atmospheric, reactor and accelerator neutrinos have established that neutrinos can oscillate and change flavour during flight \todocite. These oscillations are only possible when neutrino's have masses. The flavour neutrinos (\Pnue, \Pnum, \Pnut) are then linear expressions of the fields of at least three mass eigenstate neutrinos \Pnu$_1$, \Pnu$_2$, and \Pnu$_3$. 

The ordinary or baryonic matter described by the \SM\ describes only 5\% of the mass (energy) content of the universe. Astrophysical evidence indicated that dark matter is contributing to approximately 27\%, and dark energy to 68\% of the content of the universe. From the measurements of the temperature and polarizations anisotropies of the cosmic microwave background by the Planck experiment \todocite, the density of cold non baryonic matter is determined. Dark energy is responsible for the acceleration with the expansion of the universe. 


At the big bang matter and antimatter is assumed to be produced in equal quantities. However, it  is clear that we are surrounded by matter. So where did all the antimatter go? In 1967, Sakharov identified three mechanisms that are necessary to obtain a global matter antimatter asymmetry~\cite{Sakharov}. These mechanisms are baryon and lepton number violation, at a given moment in time there was a thermal imbalance for the interactions in the universe, and there is charge C and charge parity CP violation\footnote{The rate of a process $i\rightarrow f$ can be different from the CP-conjugate process: $\tilde i \rightarrow \tilde f$. The \SM\ includes sources of CP-violation through the residual phase of the CKM matrix. However, these could not account for the magnitude of the asymmetry observed.}.
% infor CP viol http://pdg.lbl.gov/2017/reviews/rpp2016-rev-cp-violation.pdf
The large numbers of free parameters in the \SM\ are taken as nine fermion masses, three CKM mixing angles and one CP violating phase, one EM coupling constant $g'$, one weak coupling constant $g$, one strong coupling constant $g_{\mathrm{s}}$, one QCD vacuum angle, one vacuum expectation value, and one mass of the scalar boson. This large number of free parameters lead to the expectation of a more elegant, general theory beyond the \SM. 

The hierarchy problem \todocite is related to the huge difference in energy between the weak scale and the Planck scale. The vev of the Brout-Englert-Higgs field determines the weak scale that is approximately 246 \GeV.  The radiative corrections to the scalar boson squared mass $m_{\PH}^2$, coming from its self couplings and couplings to fermions and gauge bosons, are quadratically proportional to the ultraviolet momentum cut-off $\Lambda_{\mathrm{UV}}$. This cut-off is at least equal to the energy to which the \SM\ is valid without the need of new physics. The \SM\ is valid up to the Planck mass making the correction to $m_{\PH}^2$ about thirty orders of magnitude larger than $m_{\PH}^2$. This implies that an extraordinary cancellation of terms should happen. This is also known as the naturalness problem of the \PH boson mass \todocite. 

The correction to the squared mass of the scalar boson coming from a fermion f, coupling to the scalar field $\phi$ with a coupling $\lambda_{\mathrm{f}}$ is given by
\begin{equation}
\Delta m_{\PH}^2 = -\frac{\left|\lambda_{\mathrm{f}}\right|^2}{8\pi^2}\Lambda_{\mathrm{UV}}^2, 
\end{equation}
while the correction to the mass from a scalar particle S with a mass $m_{\mathrm{S}}$, coupling to the scalar field with a Lagrangian term $-\lambda_{mathrm{S}}|\phi|^2|\mathrm{S}|^2$ is 
\begin{equation}
\Delta m_{\PH}^2 = -\frac{\left|\lambda_{\mathrm{S}}\right|^2}{16\pi^2}\left(\Lambda_{\mathrm{UV}}^2 - 2 m_{\mathrm{S}}^2 \mathrm{ln}\left(\frac{\Lambda_{\mathrm{UV}}}{m_{\mathrm{S}}}\right) + ...\right). 
\end{equation}
As one can see the correction term to $m_{\PH}^2$ is much larger than $m_{\PH}^2$ itself. By introducing BSM physic models that introduce new scalar particles at \TeV\ scale that couple to the scalar boson can cancels the $\Lambda_{\mathrm{UV}}^2$ divergence \todocite and avoid this fine-tuning. 

Also the large mass differences between the fermions related to the Yukawa couplings can go up to six order of magnitude in the case of the electron and the top quark and constitute the fermion mass hierarchy problem \todocite. 


The choice of the \SSU\ symmetry group itself  as well as the seperate treatment of the three forces included in the \SM\ raises concern. The intensity of the forces show a large disparity around the electroweak scale, but have comparable strengths at higher energies. The electromagnetic and weak forces are unified in a electroweak interaction, but the strong coupling constant does not encounter the other coupling constants at high energies. In order to reach a grand unification, the running of couplings can be modified by the addition of new particles in \BSM\ models \todocite. 


\section{An effective approach beyond the \SM: FCNC involving a top quark}
\label{sec:EFT}
The closeness of the top mass to the electroweak scale led physicist to believe that it is a sensitive probe for new physics. Its property study is therefore an important topic of the experimental program at the LHC. Several extensions of the \SM\ enhance the FCNC branching ratios and can be probed at the LHC~\cite{AguilarSaavedra:2004wm}, from which some of them are shown in \tab{tab:FCNCBRnp}. Previous searches have been performed at the Fermilab Tevatron by the CDF \cite{PhysRevLett.101.192002} and D0 \cite{Abazov:2010qk} collaborations, 
and at the LHC by the ATLAS \cite{Aad:2015uza,Aad:2015gea} and CMS \cite{Sirunyan:2017kkr,Chatrchyan:2013nwa,Khachatryan:2015att,Sirunyan:2017kkr}  collaborations\todo{check with references from TOP2017}.
\begin{table}[htbp]
	\centering
	\caption{The predicted branching ratios \BR\ for FCNC interactions involving the top quark in some  \BSM\ models~\cite{AguilarSaavedra:2004wm}: quark singlet (QS), generic two Higgs doublet model (2HDM) and the minimal super symmetric extensions to the \SM\ (MSSM);}
	\begin{tabular}{lccclccc}
		\toprule
		Process	& QS & 2HDM & MSSM &  Process	&  QS & 2HDM & MSSM\\ 
		\midrule
		$ \Ptop \rightarrow \Pup \PZ $     & $\leq 1.1 \; 10^{-4}$&$-$&$\leq 2 \; 10^{-6}$&$ \Ptop \rightarrow \Pcharm \PZ $      & $\leq 1.1 \; 10^{-4}$& $\leq 10^{-7}$& $\leq 2 \; 10^{-6}$\\
		$ \Ptop \rightarrow \Pup \Pphoton $& $\leq 7.5 \; 10^{-9}$&$-$&$\leq 2 \; 10^{-6}$&$ \Ptop \rightarrow \Pcharm \Pphoton $ & $\leq 7.5 \; 10^{-9}$& $\leq 10^{-6}$ &$\leq 2 \; 10^{-6}$\\
		$ \Ptop \rightarrow \Pup \Pgluon $ & $\leq 1.5 \; 10^{-7}$&$-$&$\leq 8 \; 10^{-5}$&$ \Ptop \rightarrow \Pcharm \Pgluon $  & $\leq 1.5 \; 10^{-7}$&  $\leq 10^{-4}$&$\leq 8 \; 10^{-5}$\\
		$ \Ptop \rightarrow \Pup \PHiggs $ & $\leq 4.1 \; 10^{-5}$&$\leq 5.5\;10^{-6}$&$\leq 10^{-5}$     &$ \Ptop \rightarrow \Pcharm \PHiggs $  & $\leq 4.1 \; 10^{-5}$& $\leq 10^{-3}$&$\leq 10^{-5}$\\
		\bottomrule
	\end{tabular} 
	\label{tab:FCNCBRnp}
\end{table}

The impact of \BSM\ models can written in a model independent way by means of an effective field theory valid up to an energy scale $\Lambda$.  The leading effects are parametrized by a set of  fully gauge symmetric dimension-6 operators that are added to the \SM\ Lagrangian and can be reduced to a minimal set of operators as discussed in~\cite{AguilarSaavedra:2008zc,AguilarSaavedra:2009mx}.  The full Lagrangian, neglecting neutrino physics, in the fully gauge symmetric case is given by 
\begin{linenomath}
	\begin{equation}
	\Lagr_{\mathrm{SM+EFT}} = \LSM + \sum \limits_{\mathrm{i}} \frac{\bar{c}_{\mathrm{i}}}{\Lambda^2}\order_{\mathrm{i}} + \order \left(\frac{1}{\Lambda^3} \right),
	\label{eq:EFTlagrangianf}
	\end{equation}
\end{linenomath}
where the Wilson coefficients $\bar{c}_{\mathrm{i}}$ depend on the considered theory and on the way that new physics couples to the \SM\ particles. Considering that $\Lambda$ is large, contributions suppressed by powers of $\Lambda$ greater than two are neglected. Moreover, all four fermion operators are omitted for the rest of this thesis. After electroweak symmetry breaking the operators induce~\cite{AguilarSaavedra:2004wm,Beneke:2000hk} both corrections to the \SM\ couplings and new interactions at tree level such as FCNC interactions. The FCNC interactions of the top quark that are not present in the \SM\ are given by
\begin{align}
\Lagr^{\Ptop}_{\mathrm{EFT}} =\frac{\sqrt{2}}{2}\sum\limits_{\Pquark = \Pup,\Pcharm} &\left[g'
\frac{\kfqt}{\Lambda} \photontensor \APtop \sigma^{\mu\nu}\left(f^{\mathrm{L}}_{\Pphoton\Pquark} P_{\mathrm{L}} + f^{\mathrm{R}}_{\Pphoton\Pquark} P_{\mathrm{R}}\right) \Pquark \right. \\
&+ \frac{g}{2\cW} \frac{\kZqt}{\Lambda} \Ztensor \APtop \sigma^{\mu\nu}\left(f^{\mathrm{L}}_{\PZ\Pquark} P_{\mathrm{L}} + f^{\mathrm{R}}_{\PZ\Pquark} P_{\mathrm{R}}\right) \Pquark \\
&+\frac{\sqrt{2}}{4\cW} \zZqt \APtop \gamma^{\mu} \left(\tilde{f}^{\mathrm{L}}_{\Pquark} P_{\mathrm{L}} + \tilde{f}^{\mathrm{R}}_{\Pquark} P_{\mathrm{R}}\right) \Pquark \PZ_{\mu} \\
&+ g_{\mathrm{S}} \frac{\kgqt}{\Lambda} \Ztensor \APtop \sigma^{\mu\nu}\left(f^{\mathrm{L}}_{\Pgluon\Pquark} P_{\mathrm{L}} + f^{\mathrm{R}}_{\Pgluon\Pquark} P_{\mathrm{R}}\right) \Pquark \Gtensor^{\mathrm{a}}\\
&+ \left. \eta_{\PHiggs\Pquark\Ptop} \APtop\left(\hat{f}^{\mathrm{L}}_{\Pquark} P_{\mathrm{L}} + \hat{f}^{\mathrm{R}}_{\Pquark} P_{\mathrm{R}}\right) \Pquark \PHiggs + \mathrm{h.c.}\right],
\label{eq:EFTlag}
\end{align}
where the the value of the \FCNC\ couplings at scale $\Lambda$ are represented by \kZqt,\kgqt,\kfqt,\zZqt, and ${ \eta_{{\PHiggs\Pquark \Ptop}}}$. These are assumed to be real and positive, with the unit of $\GeV^{-1}$ for $\kxqt$ and no unit for $\zeta_{xqt}$ and $\eta_{\mathrm{xqt}}$. In the equation $\sigma^{{\mu \nu}}$ equals to $\frac{i}{2}\left[\gamma^{{\mu}},\gamma^{\nu}\right]$,  and the left- and right-handed chirality projector operators are denoted by $P_{\mathrm{L}}$ and $P_{\mathrm{R}}$. The electromagnetic coupling constant is denoted by $g'$, the strong interaction coupling is denoted as $g_{\mathrm{s}}$, while the electroweak interaction is parametrised by the coupling constant $g$ and the electroweak mixing angle $\theta_{\mathrm{W}}$.  The complex chiral parameters are normalized according to
$ |f_{\mathrm{xq}}^{\mathrm{L}}|^2 + |f_{\mathrm{xq}}^{\mathrm{R}}|^2 = 1 $, $|\tilde{f}_{\mathrm{q}}^{\mathrm{L}}|^2 + |\tilde{f}_{\mathrm{q}}^{\mathrm{R}}|^2 = 1$, and $|\hat{f}_{\mathrm{q}}^{\mathrm{L}}|^2 + |\hat{f}_{\mathrm{q}}^{\mathrm{R}}|^2 = 1$. In the expression for $\Lagr^{\Ptop}_{\mathrm{EFT}}$, the unitary gauge is adopted on the scalar field is expanded around its vacuum expectation value with \PHiggs being the \SM\ scalar boson, and the field strength tensors of the photon \photonfield, the gluon field \Gfields, and the \PZ\ boson \Zfield\ are defined as
\begin{equation}
	\photontensor = \partial_{\mu} \mathrm{A}_{\nu} -  \partial_{\nu} \mathrm{A}_{\mu}, \;  \PZ_{\mu\nu} = \partial_{\mu} \mathrm{Z}_{\nu} -  \partial_{\nu} \mathrm{Z}_{\mu},\: \mathrm{ and } \;
	\Gtensor = \partial_{\mu} \mathrm{G}_{\nu}^{\mathrm{a}} -  \partial_{\nu}  \mathrm{G}_{\mu}^{\mathrm{a}} + g_{\mathrm{S}} f^{\mathrm{a}}_{\mathrm{bc}}   \mathrm{G}_{\mu}^{\mathrm{b}} \mathrm{G}_{\nu}^{\mathrm{c}}.
\end{equation}
Denoting the structure constant of the \Sthree\ group  as $f^{\mathrm{a}}_{\mathrm{bc}}$. Note that there are two coupling constants arising in $\Lagr^{\Ptop}_{\mathrm{EFT}}$, which is a residu of electroweak symmetry breaking. The massive \PZ\ boson will appear in both the \Zfield\ field as well as the covariant derivative, leading to an extra \PZ-vertex. 
\begin{comment}
\begin{linenomath}
	\begin{equation}
	\begin{aligned}
	\Lagr_{\mathrm{SM+EFT}} &= \sum \limits_{{\Pquark=\Pup,\Pcharm,\Ptop}} \left[ 
	\frac{\sqrt 2}{2} {g_{\mathrm{s}}} { \frac{\kgqt}{\Lambda}} \APtop \sigma^{\mu \nu} \left( {f_{{\Pgluon\Pquark}}^{\mathrm{L}}} P_{\mathrm{L}} + {f_{{\Pgluon\Pquark}}^{\mathrm{R}}}P_{\mathrm{R}}\right) \Pquark \Gtensor^{\mathrm{a}} 
	+ \frac{\sqrt 2}{2} e { \frac{\kfqt}{\Lambda}} \APtop \sigma^{\mu \nu} \left( {f_{\Pphoton \Pquark}^{\mathrm{L}}} P_{\mathrm{L}} + {f_{\Pphoton \Pquark}^{\mathrm{R}}}P_{\mathrm{R}}\right) \Pquark \photontensor  \right.\\
	&+ \frac{1}{\sqrt 2}  { \eta_{{\PHiggs\Pquark \Ptop}}} \APtop  \left( {f_{\PHiggs \Pquark }^{\mathrm{L}}} P_{\mathrm{L}} + {f_{{\PHiggs \Pquark }}}^{\mathrm{R}}P_{\mathrm{R}}\right) \Pquark \PHiggs 
	+ \frac{\sqrt 2}{4} {\frac{g}{\cW}} { \frac{\kZqt}{\Lambda}} \APtop \sigma^{\mu \nu} \left( {f_{{\PZ\Pquark}}^{\mathrm{L}}} P_{\mathrm{L}} + {f_{{\PZ\Pquark}}^{\mathrm{R}}} P_{\mathrm{R}}\right) \Pquark \Ztensor \\
	&\left. + \frac{1}{4} {\frac{g}{\cW}} { \zZqt} \APtop \gamma^{\mu} \left( {\tilde{f}_{{\PZ\Pquark}}^{\mathrm{L}}} P_{\mathrm{L}} + {\tilde{f}_{{\PZ\Pquark}}^{\mathrm{R}}}P_{\mathrm{R}}\right) \Pquark \PZ_{\mu} \right] + \mathrm{h.c.} \\
	&+ \sum \limits_{\Pquark=\Pdown,\Pstrange,\Pbottom} \left[ \frac{1}{2} {g} { \frac{\kappa_{\Ptop\PW\Pquark}}{\Lambda}} \APtop \sigma^{\mu \nu} \left( {f_{\PW \Pquark}^{\mathrm{L}}} P_{\mathrm{L}} + {f_{\PW \Pquark}^{\mathrm{R}}}P_{\mathrm{R}}\right) \Pquark \PW^+_{\mu \nu} + \frac{\sqrt 2}{4} {g} { \zWqt} \APtop \gamma^{{\mu} } \left( {\tilde{f}_{{\PW \Pquark}}^{\mathrm{L}}} P_{\mathrm{L}} + {\tilde{f}_{{\PW \Pquark}}^{\mathrm{R}}}P_{\mathrm{R}}\right) \Pquark \PW^+_{{\mu}} \right] \\ 
	&+ \mathrm{h.c.} , 
	\end{aligned}
	\label{eq:EFTlagrangianexpanded}
	\end{equation}
\end{linenomath}
where the the value of the \FCNC\ couplings at scale $\Lambda$ are represented by \kZqt,\kgqt,\kfqt,\zZqt, and ${ \eta_{{\PHiggs\Pquark \Ptop}}}$. These are assumed to be real and positive, with the unit of $\GeV^{-1}$ for $\kxqt$ and no unit for $\zeta_{xqt}$ and $\eta_{\mathrm{xqt}}$. In the equation $\sigma^{{\mu \nu}}$ equals to $\frac{i}{2}\left[\gamma^{{\mu}},\gamma^{\nu}\right]$,  and the left- and right-handed chirality projector operators are denoted by $P_{\mathrm{L}}$ and $P_{\mathrm{R}}$. The electromagnetic coupling constant is denoted by $e$, the strong interaction coupling is denoted as $g_{\mathrm{s}}$, while the electroweak interaction is parametrised by the coupling constant $g$ and the electroweak mixing angle $\theta_{\mathrm{W}}$.  The complex chiral parameters and are assumed to be real  and  fulfil the relation 
$ \left(|f_{\mathrm{xq}}^{\mathrm{L}}|^2 + |f_{\mathrm{xq}}^{\mathrm{R}}|^2 \right)= 1 $ and $\left(|\tilde{f}_{\mathrm{xq}}^{\mathrm{L}}|^2 + |\tilde{f}_{\mathrm{xq}}^{\mathrm{R}}|^2 \right)= 1$.
% The limitations of the electro weak broken phase approach are summarized in \cite{Durieux:2014xla}. 

\end{comment}

\section{Experimental constraints on top-FCNC}
\label{sec:ExpConstr}
Experiments commonly put limits on the branching ratio's which allow an easier interpretation across different EFT models as
\begin{equation}
	\BR(\Ptop \rightarrow \Pquark\mathrm{X}) = \frac{\delta^2_{\Ptop \mathrm{X}\Pquark}\Gamma_{\Ptop \rightarrow \Pquark\mathrm{X}}}{\Gamma_{\Ptop}},
\end{equation}
where $\Gamma_{\Ptop \rightarrow \Pquark\mathrm{X}}$ represents the \FCNC\ decay width\footnote{The decay width of a certain process represents the probability per unit time that a particle will decay. The total decay width, defined as all possible decay widths of a particle, is inversely proportional to its lifetime. } for a coupling strength $\delta^2_{\Ptop \mathrm{X}\Pquark}=1$, and $\Gamma_{\Ptop}$ the full decay width of the top quark. In the \SM, supposing a top quark mass of 172.5 \GeV, the full width becomes $\Gamma_{\Ptop}^{\mathrm{SM}} = 1.32$ \GeV~\cite{Gao:2012ja}. 

% top fcnc vertex in feynman diagram -> matrix element maal kappa --> cross sectie maal kappa^2 --> cross sectie maal BR (lineair)


Searches for top-FCNC usually adopt a search strategy depending on the experimental setup and the FCNC interaction of interest,  looking either for \FCNC\ interactions in the production of a single top quark or in its decay for top pair interactions. In \fig{fig:Feynman}, these two cases are shown for the \tZq\ vertex.  \\

\begin{figure}[hbtp]
	\centering
	\subbottom[]{
			\begin{fmffile}{singletop}
		\begin{fmfgraph*}(160,40) % width - height
\fmfleft{i1,i2} 
\fmfright{o1,o2}
 \fmf{fermion}{i1,v1,v2,o2}
  \fmf{boson}{v2,o1}
   \fmf{gluon}{i2,v1}
   \fmflabel{\Pgluon}{i2}
   \fmflabel{\Pup,\Pcharm}{i1}
   	\fmfv{label= ,decor.shape=circle,decor.filled=shaded,decor,size=0.5thick, label.angle=-90}{v2}
  % \fmflabel{\Pup,\Pcharm}{i1}
   \fmflabel{\PZ}{o1}
   \fmflabel{\Ptop}{o2}
    \fmf{fermion,label=\Pup,,\Pcharm,label.dist=10}{v1,v2}
      \end{fmfgraph*}
\end{fmffile}
}
\hspace*{1cm}
\subbottom[]{
	\begin{fmffile}{toppair}
	\begin{fmfgraph*}(160,40) % width - height
		\fmfleft{i1,i2} 
		\fmfright{o1,o2,o3,o4}
		\fmflabel{\Pgluon}{i1}
		\fmflabel{\Pgluon}{i2}
		\fmflabel{\APup,\APcharm}{o1}
		\fmflabel{\PZ}{o2}
		\fmflabel{\PWp}{o3}
		\fmflabel{\Pbottom}{o4}
		\fmf{gluon}{i1,v1,i2}
		\fmf{gluon,label=\Pgluon,label.dist=10}{v1,v2}
		\fmf{fermion}{o1,v4,v2,v3,o4}
		\fmffreeze
		\fmf{boson}{v3,o3}
		\fmf{boson}{v4,o2}
		\fmfv{label= ,decor.shape=circle,decor.filled=shaded,decor,size=0.2thick, label.angle=-90}{v4}
		 \fmf{fermion,label=\Ptop,label.dist=5}{v2,v3}
		 \fmf{fermion,label=\APtop,label.dist=-20}{v4,v2}
		\fmflabel{}{v3}
		\fmflabel{}{v2}
		\fmflabel{}{v1}
	\end{fmfgraph*}
\end{fmffile}
}
\caption{Feynman diagrams for the \tZq\ \FCNC\ interaction, where the FCNC interaction is indicated with the shaded dot. Left: single top production through an \FCNC\ interaction. Right: top pair production with an \FCNC\ induced decay. }
\label{fig:Feynman}
\end{figure}


The observation of top-FCNC interactions has yet to come and experiments have so far only been able to put upper bounds on the branching ratios. An overview of the best current limits is given in \tab{tab:FCNClimits} \todo{update  after TOP2017}. In \fig{fig:fcncupperlimits} \todo{update figure after TOP2017} a comparison is shown between the current best limits set by ATLAS and CMS with respect to several \BSM\ model benchmark predictions. From there one can see that \FCNC\ searches involving a \PZ\ or \PHiggs\ boson are close to excluding or confirming several \BSM\ theories.
\begin{table}[htbp]
	\centering
	\caption{The four forces of nature and their characteristics.}
	\begin{tabular}{lcccc}
		\toprule
		Process & Search mode & Observed \BR & Expected \BR & Reference \\ 
		\midrule
			\bottomrule
	\end{tabular} 
	\label{tab:FCNClimits}
\end{table}

\begin{figure}[htbp]
	\centering
	\includegraphics[width=0.7\linewidth]{1_Introduction/Figures/fcnc_upperlimits}
	\caption{Current best limits set by CMS and ATLAS for top-FCNC interactions.}
	\label{fig:fcncupperlimits}
\end{figure}







\chapter{Experimental set-up}
%\epigraph{The LHC rose from the dead on Easter. I've heard thats been done before... There is a place in human knowledge where here be dragons. That is the place where the ship that is the LHC  will be going.}
%	{\textit{Don Lincoln, tedX}}


A key objective of the Large Hadron Collider (LHC) was the search for the Brout-Englert-Higgs boson. The Large  Electron Positron (LEP)~\cite{Myers:226776} and Tevatron~\cite{1748-0221-6-08-T08001} experiments established that the mass of the scalar boson had to be larger than 114 \GeV~\cite{Barate:2003sz,Herner:2016woc}, and smaller than approximate 1 \TeV\ due to unitarity and perturbativity constraints~\cite{Djouadi:2005gi}. On top of this, the search for new physics such as supersymmetry or the understanding of dark matter were part of the motivation for building the LHC. 
Since the start of its operation, the LHC is pushing the boundaries of the Standard Model, putting the most stringent limits on physics beyond the Standard Model as well as precision measurements of the parameters of the Standard Model. A milestone of the LHC is the discovery of the scalar boson in 2012 by the two largest experiments at the LHC~\cite{Chatrchyan:2012xdj,Aad:2012tfa}.

This chapter is dedicated to the experimental set-up of the LHC and the Compact  Muon Solenoid (CMS) experiment. \Sec{sec:LHC} describes the LHC and its acceleration process for protons to reach their design energies. The CMS experiment and its components are presented in \Sec{sec:CMS}. The upgrades performed during the long shutdown in 2013 are discussed in \Sec{sec:Phase1}. The data acquisition of CMS is presented in \Sec{sec:DAQ}, while the CMS computing model is shown in \Sec{sec:Computing}.


\section{The Large Hadron Collider}
\label{sec:LHC}
The LHC has started its era of cutting edge science on 10 September 2008~\cite{LHC:2008} after approval by the European Organisation of Nuclear Research (CERN) in 1995~\cite{Pettersson:291782}. Installed in the previous LEP tunnel, the LHC consists of a 26.7~\km\ quasi ring, that is installed between 45 and 170~\m\ under the French-Swiss border amidst Cessy (France) and Meyrin (Switzerland). Built to study rare physics phenomena at high energies, the LHC  can accelerate mainly two types of particles, protons and lead ions Pb$^{45+}$, and provides collisions at four interaction points, where the particle bunches are crossing. Experiments for studying the collisions are installed at each interaction point. 

 As can be seen in \fig{fig:LHCchain}, the LHC is the last element in a chain that creates, injects and accelerates protons. The starting point is the ionisation of hydrogen, creating protons that are injected in a linear accelerator (LINAC 2). Here, the protons obtain an energy of 50~\MeV. They continue to the Proton Synchrotron Booster (PSB or Booster), where the packs of protons are accelerated to 1.4~\GeV\ and each pack is split up in twelve bunches with 25 or 50~\nano\s\ spacing. The Proton Synchrotron (PS) then increases their energy to 25~\GeV\ before the Super Proton Synchrotron (SPS) increases the proton energy up to 450~\GeV. Each accelerator ring expands in radius in order to reduce the energy loss of the protons by synchrotron radiation\footnote{This energy loss is proportional to the fourth power of the proton energy and inversely proportional to the bending radius.}. Furthermore, the magnets responsible for the bending of the proton trajectories have to be strong enough to sustain the higher proton energy. Ultimately, the proton bunches are injected into opposite directions into the LHC, where they are accelerated to 3.5 \TeV\ (in 2010 and 2011), 4~\TeV\ (in 2012 and 2013) or 6.5~\TeV\ (in 2015-2017)~\cite{Wenninger:2254678}. 
  \begin{figure}[h]
 	\centering
 	\includegraphics[width=1.\textwidth]{2_ExperimentalSetup/Figures/CCC-v2016}
 	\caption{Schematic representation of the accelerator complex at CERN~\cite{DeMelis:2197559}. The LHC (dark blue) is the last element in chain of accelerators. Protons are successively accelerated by LINAC 2, the Booster, the Proton Synchrotron (PS) and the Super Proton Synchrotron (SPS) before entering the LHC.}
 	\label{fig:LHCchain}
 \end{figure}

In \fig{fig:lhcschedule} the LHC programme is shown. the first data collisions, so-called Run 1 period, lasted from 2008 until 16 February 2013 after which  the CERN accelerator complex shut down for two years of planned maintenance and consolidation during so-called long shutdown 1 (LS1). On 23 March 2015, the new data taking period known as Run 2 started. There was a brief end of the year extended technical stop (EYETS) at the end of 2016. The main activities carried out during the EYETS were the maintenance of systems such as the cryogenics, the cooling, electrical systems, as well as a de-cabling and cabling campaign on the SPS~\cite{MurilloQuijada:2017agx}. Run 2 will last until July 2018 when the  long shutdown 2 (LS2) will begin for 2 years. The main goal of this shutdown is the LHC injectors upgrade (LIU), but also maintenance and consolidation will be performed. Furthermore, preparations for the High Luminosity LHC (HL-LHC), which will start in 2024, will be done. More information about phase 1 upgrades during the LS1 and the EYETS in 2016 for CMS is given in \Sec{sec:Phase1}.\\ %https://home.cern/cern-people/updates/2017/01/eyets-report-so-far-so-good\
 Before the start of the LHC in 2010, the previous energy record was held by the Tevatron collider at Fermilab, colliding protons with antiprotons at \com = 1.96~\TeV.  When completely filled, the LHC nominally contains 2220 bunches in Run 2, compared to 1380 in Run 1. % (design: 2200). 
%At full intensity, it would have nearly 2800 bunches but this is limited due to a problem with SPS in 2016.
\begin{figure}[htbp]
	\centering
	\includegraphics[width=1.\linewidth]{2_ExperimentalSetup/Figures/lhc_schedule}
	\caption{The HL-LHC timeline. Figure taken from \cite{Antonella:1975962}.}
	\label{fig:lhcschedule}
\end{figure}


Inside the LHC ring~\cite{Bruning:782076}, the protons are accelerated by  means of radio frequency cavities, while 1232 dipole magnets of approximately 15~\m\ long, each weighing 35~\tonne\, ensure the deflection of the beams.  The two proton beams circulate in opposite direction in separate pipes inside of the magnet. Through the use of a strong electric current in the coils of the magnet, magnetic fields are generated and cause the protons to bend in the required orbits. In order for the coil to become superconducting and able to produce a strong magnetic field of 8.3~\tesla, the magnet structure is surrounded by a vessel. This vessel is filled with liquid Helium making it possible to cool down the magnet to 1.9~\kelvin. In order to get more focussed and stabilised proton beams, additional higher-order multipole and corrector magnets are placed along the LHC beam line.

  

The LHC is home to seven experiments, each located at an interaction point: 
\begin{itemize}
	\item A Toroidal LHC ApparatuS (ATLAS)~\cite{Aad:2008zzm} and the Compact Muon Solenoid (CMS)~\cite{Chatrchyan:2008aa} experiments are the two general purpose detectors at the LHC. They both have a hermetic, cylindrical structure and were designed to search for new physics phenomena along with precision measurements of the Standard Model. The existence of two distinct experiments allows cross-confirmation of any discovery. 
	\item A Large Ion Collider Experiment (ALICE)~\cite{Aamodt:2008zz} and the LHC Beauty (LHCb)~\cite{Alves:2008zz} experiments are focusing on specific phenomena. ALICE studies strongly interacting matter at extreme energy densities where a quark-gluon plasma forms in heavy ion collisions (Pb-Pb or p-Pb). LHCb searches for differences between matter and antimatter with the focus on \Pbottom\ quark physics.
	\item The forward LHC (LHCf)~\cite{Bongi:2010zz} and the TOTal cross section, Elastic scattering and diffraction dissociation Measurement (TOTEM)~\cite{Anelli:2008zza} experiments are two smaller experiments that focus on head-on collisions. LHCf consists of two parts placed before and after ATLAS and studies particles created at very small angles. TOTEM is placed in the same cavern as CMS and measures the total proton-proton cross section and studies elastic and diffractive scattering.  
		\item The Monopoles and Exotics Detector At the LHC (MoEDAL)~\cite{Acharya:2014nyr} experiment is situated near LHCb and tries to find magnetic monopoles. 
\end{itemize}



 For the enhancement of the exploration of rare events and thus enhancing the number of collisions, high beam energies as well as high beam intensities are required. The luminosity~\cite{Gillies:1997001} is a measure of the number of collisions that can be produced in a detector per square meter and per second and is the key role player in the enhancement of the beam intensity. The LHC collisions create a number of events per second given by
\begin{equation}\label{eq:NbEv}
\centering
N_{\mathrm{event}} = L \sigma_{\mathrm{event}}, 
\end{equation}
where $\sigma_{\mathrm{event}}$ is the cross section of the process of interest and $L$ the machine  instantaneous luminosity. This luminosity depends only on the beam parameters and is for a Gaussian beam expressed as 
\begin{equation}\label{eq:Lumi}
	L = \frac{1}{4\pi}\textcolor{blue}{N_{\mathrm{b}} n_{\mathrm{b}} f_{rev}}\textcolor{red}{\frac{N_{\mathrm{b}}}{\epsilon_n }}\textcolor{darkgreen}{\left(1+\left(\frac{\theta_c \sigma_z}{2\sigma^*} \right)^2\right)^{-\frac{1}{2}}} \frac{\gamma_{\mathrm{r}}}{ \beta^*}.
\end{equation}
The number of particles per bunch is expressed by $N_{\mathrm{b}}$, while $n_{\mathrm{b}}$ is the number of bunches per beam, $f_{\mathrm{rev}}$ the revolution frequency, $\gamma_{\mathrm{r}}$ the relativistic gamma factor, $\epsilon_n$ the normalized transverse beam emittance - a quality for the confinement of the beam  , $\beta^*$ the beta function at the collision point - a measurement for the width of the beam, $\theta_c$ the angle between two beams at the interaction point, $\sigma_{\mathrm{z}}$ the mean length of one bunch, and $\sigma^*$ the mean height of one bunch. In \eq{eq:Lumi}, the blue part represents the stream of particles, the red part  the brilliance, and the green part  the geometric reduction factor due to the crossing angle at the interaction point.

The peak design luminosity for the LHC is $10^{34}\:\centi\m^{-2}\s^{-1}$, which leads to about 1 billion proton interactions per second. In 2016, the LHC was around 10\% above this design luminosity~\cite{Harriet:2212301}. The luminosity is not a constant in time since it diminishes due to collisions between the beams, and the interaction of the protons and the particle gas that is trapped in the centre of the vacuum tubes due to the magnetic field. The diffusion of the beam degrades the emittance and therefore also the luminosity. For this reason, the mean lifetime of a beam inside the LHC is around 15~\hour. The integrated luminosity - the luminosity provided in a certain time range - recorded by CMS over the year 2016 is given in \fig{fig:IntLumi}. In Run 2, the peak luminosity is 13-17$ \times 10^{34}\; \centi\m^{-2}\s^{-1}$ compared to 7.7$ \times 10^{33}\; \centi\m^{-2}\s^{-1}$ in Run~1. The recorded luminosity is validated for physics analysis keeping 35.9~\fbinv\ during 2016 data taking. 
% https://science.energy.gov/~/media/hep/hepap/pdf/201512/CMSRestart_Rakness_HEPAP_20151210.pdf
%https://indico.in2p3.fr/event/10819/contributions/3365/attachments/2415/2966/6_Lamont_LHC-moriond-march15-v2.pdf
 \begin{figure}[htbp]
 	\centering
%	\includegraphics[width=0.7\textwidth]{2_ExperimentalSetup/Figures/int_lumi_per_day_cumulative_pp_2016.pdf}
	\includegraphics[width=0.7\textwidth]{2_ExperimentalSetup/Figures/int_lumi_per_day_cumulative_pp_2016_Golden_23Sep-PromEraH_Morion}
	\caption{Cumulative offline luminosity measured versus day delivered by the LHC  (blue), and recorded by CMS (orange), and certified as good for physics analysis during stable beams (light orange) during stable beams and for proton collisions at 13 TeV centre-of-mass energy in 2016.~\cite{LumiWiki}. }
	\label{fig:IntLumi}
%	(from https://twiki.cern.ch/twiki/bin/view/CMSPublic/LumiPublicResults#Online_Luminosity_AN2 )
%  https://twiki.cern.ch/twiki/bin/view/CMSPublic/DataQuality#2016_Proton_Proton_Collisions
\end{figure}
 
Multiple proton-proton interactions can occur during one  bunch crossing, referred to as \pu. On average, the number of \pu\ events is proportional to the luminosity times the total inelastic proton-proton cross section. In 2016,  an average of about 27 of \pu\ interactions has been observed in 13 \TeV\ proton collisions at the interaction point of CMS. For 2012, this number was about 21 \pu\ interactions for 8 \TeV\ collisions.



%CMS collaboration http://cds.cern.ch/record/2195940?ln=en map

\begin{comment}
% CMS Luminosity Measurement for the 2016 Data Taking Period`(CADI entry LUM-17-001)
% http://cms.cern.ch/iCMS/analysisadmin/cadi?ancode=LUM-17-001

CERN ACC 2017 0007 --> beam energy uncertainty

\end{comment}
\section{The Compact Muon Solenoid}
\subsection{CMS coordinate system}
\subsection{Inner trackking system and operations}
\subsection{Electromagnetic calorimeter}
\subsection{Hadronic calorimeter}
\subsection{Muon system}
\subsection{Data acquisition}
\subsection{CMS computing model}


\chapter{Analysis techniques}

\section{Hadron collisions at high energies}
In hadron collisions at as sufficiently high momentum transfer, all partons can be approximated as free  making it possible to treat hadron-hadron scattering as a single parton-parton interaction. The momentum of the parton can then be expresses as a fraction of the hadron momentum 
\begin{equation}
 \vec{p}_{\mathrm{parton}} = x \vec{p}_{\mathrm{hadron}}, 
\end{equation}
where $x$ is referred to as the Bj\"orken scaling variable. The interaction $\Pproton_{\mathrm{A}} \Pproton_{\mathrm{B}} \rightarrow \mathrm{X}$ can then be factorised in terms of partonic cross sections $\hat{\sigma}_{\mathrm{ij}\rightarrow\mathrm{X}}$~\cite{Collins:1989gx}
\begin{equation}
 \sigma_{\mathrm{p}_{\mathrm{A}}\mathrm{p}_{\mathrm{B}}\rightarrow\mathrm{X}} = \sum \limits_{\mathrm{ij}} \iint dx_1 dx_2  \: f_{\mathrm{i}}^{\mathrm{A}}(x_{\mathrm{1}},Q^2)f_{\mathrm{j}}^{\mathrm{B}}(x_{\mathrm{2}},Q^2) {d\hat{\sigma}_{\mathrm{ij}\rightarrow\mathrm{X}}}, 
 \label{eq:cross}
 \end{equation}
where i and j are the partons resolved from protons A and B,  $f_{\mathrm{i}}(x_{\mathrm{i}},Q^2)$ the parton density functions (PDF), and $Q^2$ the factorisation scale more commonly denoted as \muF. The factorisation scale is the scale at which the hadronic interaction can be expressed as a product of the partonic cross section and the process independent PDF. In \fig{fig:factoscale}, the kinematic regions in $x$ and \muF\ are shown for fixed target and collider experiments.
\begin{figure}
	\centering
	\includegraphics[width=0.5\linewidth]{3_Analysis_techniques/Figures/factoscale}
	\caption{Kinematic regions in momentum fraction $x$ and factorisation scale $Q^2$ probed by fixed-target and collider experiments. Some of the final states accessible at the LHC are indicated in the appropriate regions, where $y$ is the rapidity. In this figure, the incoming partons have $x_{1,2} = (M/14 \TeV)e^{\pm y}$ with $Q = M$ where $M$ is the mass of the state shown in blue in the figure. For example, exclusive J$/\psi$ and $\Upsilon$ production at high $|y|$ at the LHC may probe the gluon PDF down to $x \sim  10^{-5}$. Figure taken from \cite{PDG}.}
	\label{fig:factoscale}
\end{figure}




 The parton density functions (PDF)~\cite{Placakyte:2011az,Ball2015,Butterworth:2015oua} give the momentum distribution of the proton amongst its partons at an energy scale \muF.  
  These function can not be determined from first principles and have to obtained from global fits to data. The PDFs are obtained from measurements on deep inelastic scattering using lepton-proton collision by the HERA collider~\cite{Abramowicz:1998ii}, supplemented with proton-antiproton collisions from Tevatron at Fermi lab~\cite{Holmes:2011ey}, and proton collision data from the ATLAS, CMS and LHCb collaborations at the LHC (Run 1)~\cite{Rojo:2015acz}. These measurements are included in global PDF sets known as the \texttt{PDF4LHC} recommendation~\cite{Butterworth:2015oua}. From their measurement at scale \muF\ these PDFs can be extrapolated using the DGLAP equations \todocite. The PDFs are used to calculate the cross section of a certain process and are therefore used as input for the Monte Carlo generators used to make the simulated data samples at the LHC. 
%https://amva4newphysics.wordpress.com/2016/03/10/the-inner-life-of-protons-and-artificial-neural-networks/
In the framework of this thesis, the NLO \texttt{PDF4LHC}15\_100 set is used. This set is an envelope of three sets, \texttt{CT14}, \texttt{MMHT2014} and \texttt{NNPDF3.0}~\cite{Butterworth:2015oua}. In \fig{fig:nnpdf30} the dependency of the PDFs on the momentum fraction $x$ is shown for the \texttt{NNPDF3.0} set on hadronic scale ($\muF^2 = (10\GeV)^2$ and LHC scale ($\muF^2 = (10^4\GeV)^2$. For most values of the momentum fraction, the gluon density dominates, meaning that it is easier to probe muons than the quarks. For $x$ close to one, the parton densities of the up and down quarks (the valence quarks of the proton) dominate over the gluon density. The charm, anti-up, and anti-down quarks have lower densities in general since those are sea quarks which originate in the proton only through gluon splitting. 
The resolution scale $Q^2$ is typically taken to be the energy scale of the collision. For the top quark pair production a scale of $Q^2=(350\: \GeV)^2$ is chosen, meaning that the centre-of-mass energy of the hard interaction is about twice the top quark mass.
\begin{figure}[htbp]
	\centering
	\includegraphics[width=0.7\linewidth]{3_Analysis_techniques/Figures/NNPDF30}
	\caption{The momentum fraction $x$ times the parton distribution functions $f(x)$, where $f=\Pup_{\mathrm{v}}, \Pdown_{\mathrm{v}} ,\APup,\APdown,\Pstrange,\Pcharm,$ or \Pgluon\ as function of the momentum fraction obtained in the NNLO \texttt{NNPDF3.0} global analysis at factorisation scales $\mu^2 = 10 \: \GeV^2$ (left) and $\mu^2=10^4 \: \GeV^2$ (right), with $\alpha_{\mathrm{S}}(M^2_{\PZ}) = 0.118$. The gluon PDF has been scaled down by a factor of 0.1. Figure taken from \cite{PDG}.}
	%http://pdg.lbl.gov/2017/reviews/rpp2016-rev-structure-functions.pdf
	% The higher value of the resolution scale $Q^2$, the smaller distances that are probed in the proton.
	\label{fig:nnpdf30}
\end{figure}
The uncertainty on the parton distributions is evaluated using the Hessian technique~\cite{Pumplin:2001ct}, where a matrix with a dimension identical to the number of free parameters needs to be diagonalised. In the case of \texttt{PDF4LHC}15\_100 set, this translates into 100 orthonormal eigenvectors and 200 variations of the PDF parameters in the plus and minus direction. 
%https://www.hep.ucl.ac.uk/pdf4lhc/LesHouches2016-PDF4LHC.pdf
%https://indico.cern.ch/event/525605/contributions/2152733/attachments/1267702/1877336/TOP_PAG_3_05_16_PDFs.pdf

At high energies divergences can appear from quantum fluctuations. For the theory still to be able to describe the experimental regime, a renormalization scale \muR\ is used to redefine physical quantities A consequence of this method is that the coupling constants will run as function of \muR. Beyond this scale, the high energy effects such as the loop corrections to propagators (self energy) are absorbed in the physical quantities through a renormalization of the fields. In particular the running behaviour of the strong coupling constant\footnote{The strong coupling constant is defined as $\alpha_{\mathrm{S}} = \frac{g_\mathrm{S}^2}{4\pi}$. } $\alpha_{\mathrm{S}}$ is found to be 
\begin{equation}
	\alpha_{\mathrm{S}} = \frac{\alpha_{\mathrm{S}}(\mu_0^2)}{1 + \alpha_{\mathrm{S}}(\mu_0^2) \frac{33 - 2 n_{\mathrm{f}}}{12 \pi}\mathrm{ln}\left(\frac{|\muR^2|}{\mu_0^2}\right)}, 
	\label{eq:couplingstrength}
\end{equation}
with $n_{\mathrm{f}}$ the number of quarks and $\mu_0$ the reference scale on which the coupling is known. The current world average of the strong coupling constant at the \PZ boson mass is $\alpha_{\mathrm{S}}(\muF = \mZ) = 0.1181 \pm 0.0011$~\cite{PDG}. From  \eq{eq:couplingstrength} one can see easily that the coupling strength decreases with increasing renormalization scale, this known as asymptotic freedom. Additionally, following the behaviour of $\alpha_{\mathrm{S}}(\muR^2)$, a limit $\Lambda_{\mathrm{QCD}} \approx 200 \: \MeV$ is found for which $\alpha_{\mathrm{S}}$ becomes larger than one. Under this limit, the perturbative calculations of observables can no longer be done.
% Mandl and shaw pagina 352!



%The cross section $\sigma$ of scattering process with a flux\footnote{This entity is more commonly referred to as Luminosity.} $\lumi= \rho v$ of incoming particles with particle density $\rho$ and velocity $v$ is defined as the number of interactions per unit density ($\rho=1$)\footnote{The cross section is usual expressed in barn, $1b = 10^{-28}\m^2$. The number of interactions per time is given by $\frac{dN}{dt} = \lumi \sigma$}. 
Cross sections be written in terms of interacting vertices contributing to the matrix element (ME) originating from elements of a perturbative series~\cite{Mandl:1236742}, allowing them to be expanded as a power series of the coupling constant $\alpha$ 
\begin{equation}
 \sigma  = \sigma_{\mathrm{LO}} \left(1 + \left(\frac{\alpha}{2\pi}\right)\sigma_1  + \left(\frac{\alpha}{2\pi}\right)^2\sigma_2 + ...\right).
\end{equation}
Leading order (LO) accuracy contains the minimal amount of vertices in the process, then depending on where the series is cut off one speaks of next-to-leading order (NLO), or next-to-next-to-leading order (NNLO) accuracy in $\alpha$. Predictions including higher order correction tend to be less affected by theoretical uncertainties originating from a variation of the chosen renormalization and factorisation scales. 
% zie thesis matthias p 21 bovenaan

\section{Event generation}
In order to compare reconstructed data with theoretical predictions, collision events are generated and passed through a simulation of the CMS detector and an emulation of its readout. For the detector simulation, a so-called Full Simulation package~\cite{1742-6596-396-2-022003,1742-6596-664-7-072022}  based on the \Geant4 toolkit~\cite{AGOSTINELLI2003250} is employed. It allows a detailed simulation of the interactions of the particles with the detector material. 
\subsection{Fundamentals of simulating a proton collision}
The procedure of to generate $\Pproton\Pproton \rightarrow \mathrm{X}$ events can be subdivided into sequential steps~\cite{Seymour:2013ega,Sjostrand:2009ad,Hoche:2014rga}, as shown in \fig{fig:ppcollision}.
\begin{figure}[htbp]
	\centering
	\includegraphics[width=1.\linewidth]{3_Analysis_techniques/Figures/MCeventwithlegend}
	\caption{Sketch of a hadron collision as simulated by a Monte-Carlo event generator. The red blob in the centre represents the hard collision, surrounded by a tree-like structure representing Bremsstrahlung as simulated by parton showers. The purple blob indicates a secondary hard scattering event. Parton-to-hadron transitions are represented by light green blobs, dark green blobs indicate hadron decays, while yellow lines signal soft photon radiation. Figure taken from~\cite{Hoche:2014rga}.}
	\label{fig:ppcollision}
\end{figure}

The interaction of two incoming protons is often soft and elastic leading to events that are not interesting in the framework of this thesis. More intriguing are the hard interaction between two partons from the incoming protons. The matrix elements   of a hard scattering process of interest is the starting point of the generation of events. Monte Carlo techniques are used to sample the corresponding cross section integral and the resulting sample of events reflect the probability distribution of a process over its final state phase space. After obtaining the sample of events of the hard interaction, a parton shower (PS) program is used to simulate the hadronisation of final state partons into hadrons which then  decay further. Additionally, radiation of soft gluons or quarks from initial or final state partons is simulated. These are respectively referred to as initial state radiation (ISR) or final state radiation (FSR). Contributions from soft secondary interactions, the so-called underlying event (UE), and colour reconnection effects are also taken into account. \todo{Should I add more details?}
A brief overview of the employed programs used for the event generation of the signal and main background processes used in the search presented in the thesis are given in \Sec{sec:programs}.

\subsection{Programs for event generation}
\label{sec:programs}
The \texttt{FEYNRULES} package~\cite{Alloul:2013bka} allows the calculation of  the Feynman rules in momentum space for any quantum field theory model. By use of a Lagrangian, the set of Feynman rules associated with this Lagrangian are calculated. Via the \UFO\ (UFO)~\cite{Degrande:2011ua} the results are then passed to matrix element generators. 


The \MG\  program~\cite{Alwall:2011uj} is used to interpret the physics model and calculate the corresponding Feynman diagrams and matrix elements. After this, \ME~\cite{Mangano:2006rw} is used to calculate the corresponding partons. These generated parton configurations are then merged with \Pythia~\cite{Sjostrand2015159,Sjostrand:2006za,Sjostrand:2014zea} parton showers using the MLM merging scheme~\cite{Alwall:2007fs}. 

The \aMCMG\ program~\cite{Alwall:2014hca} combines the LO \MG~\cite{Alwall:2011uj} and the \aMC\ program into a common framework. This combination supports the generation of samples at LO or next to NLO together with a dedicated matching to parton showers  using the MLM~\cite{Alwall:2007fs} or FXFX~\cite{Frederix:2012ps} schemes respectively. The FXFX scheme produces a certain fraction of events with negative weights originating from the subtraction of amplitudes that contain additional emissions from the NLO matrix element to prevent double-counting.
%or MC$@$NLO~\cite{Frederix:2012ps}  \todo{MC$@$NLO voor ME generator }



The \Powheg\ box (versions 1,2)~\cite{Alioli2010,1126-6708-2009-09-111,1126-6708-2007-11-070,Alioli:2010xd,Frixione:2007vw,Nason:2004rx} contains predefined implementations of various processes at NLO. It applies the \Powheg\ method for ME- to PS- matching, where the hardest radiation generated from the ME has priority over subsequent PS emission to remove the overlap with the PS simulation.

The \JHU\ generator (version 7.02)~\cite{Gritsan:2016hjl,Anderson:2013afp,Bolognesi:2012mm,Gao:2010qx} is used to generate the parton level information including full spin and polarization correlations. It is commonly used for studying the spin an parity properties of new resonances such as $\mathrm{ab}\rightarrow\mathrm{X}\rightarrow \mathrm{VV}$, where $\mathrm{V} = \PZ, \PW, \Pphoton)$. 

The generation of events from processes involving the production and decay of resonances creates a computational heavy load, especially at NLO. The narrow width approximation the resonant particle is assumed to be on-shell. This makes the production and decay amplitude factorize, allowing to perform the simulation of the production and decay of heavy resonances like top quarks or Higgs bosons to be performed in separate steps. The \MS\ program~\cite{Artoisenet:2012st} extends this approach and accounts for off-shell effects through a partial reweighting of the events. Additionally, spin correlation effects between production and decay products are taken into account. 

The \Pythia\ program (versions 6,8)~\cite{Sjostrand2015159,Sjostrand:2006za,Sjostrand:2014zea} generates events of various processes at LO. Usually in the analysis, it is however only used for its PS simulation and it is interfaced with other LO and NLO event generators to perform subsequent parton showering, hadronisation, and simulation of the underlying event.  In this thesis the underlying event tunes~\cite{Khachatryan2016}  are the CUETP8M2T4, CUETP8M1 and CUETP8M2. 





The detector response is simulated via the \Geant 4~\cite{AGOSTINELLI2003250} program. This program tracks the particles through the detector material via a detailed description of the detector and generates several hits throughout several sensitive layers. 
In addition, the response of the detector electronics to these hits are simulated. 


\subsection{Generating FCNC top-Z interactions}
The FCNC processes are generated by interfacing the Lagrangian in \eq{eq:EFTlag} with \aMCMG\ by means of the \FR\ package and its  \UFO\ format.  The complex chiral parameters are arbitrary chosen to be $f^{\mathrm{L}}_{\mathrm{X}\Pquark} = 0$ \todo{Why LH and not RH?} and  $f^{\mathrm{R}}_{\mathrm{X}\Pquark} = 1$. The signal rates are estimated by use of the \aMCMG\ program for estimating the partial widths. The anomalous couplings are left free to float for this estimation, and only one coupling allowed to be non-vanishing at a time. The results are presented in \tab{tab:partialwidths}.
\begin{table}[htbp]
	\centering
	\caption{Leading order partial widths related to the anomalous decay modes of the top quark, where the new physics scale $\Lambda$ is given in \GeV.}
	\begin{tabular}{ccll}
		\toprule
		Anomalous coupling & vertex & \multicolumn{2}{c}{Partial decay width  (\GeV) }\\ 
		\midrule
		\multirow{2}{*}{\kgqtl} & \Ptop\Pgluon\Pup      &  3.665220 $10^{5}$   & $\left( \kappa_{\Ptop\Pgluon\Pup} / \Lambda \right)^2$ \\
		                    & \Ptop\Pgluon\Pcharm       &  3.664620 $10^{5}$   & $\left( \kappa_{\Ptop\Pgluon\Pcharm} / \Lambda \right)^2$ \\
	    \multirow{2}{*}{\kfqtl} & \Ptop\Pphoton\Pup     &  1.989066 $10^{4}$   & $\left( \kappa_{\Ptop\Pphoton\Pup} / \Lambda \right)^2$ \\
		                    & \Ptop\Pphoton\Pcharm      &  1.988904 $10^{4}$   & $\left( \kappa_{\Ptop\Pphoton\Pcharm} / \Lambda \right)^2$    \\
		\multirow{2}{*}{\kZqtl} & \Ptop\PZ\Pup          &  1.637005 $10^4$     & $\left( \kappa_{\Ptop\PZ\Pup} / \Lambda \right)^2$     \\
		                    & \Ptop\PZ\Pcharm           &   1.636554 $10^4$    & $\left( \kappa_{\Ptop\PZ\Pcharm} / \Lambda \right)^2$  \\
		\multirow{2}{*}{\zZqt} & \Ptop\PZ\Pup           &   1.685134 $10^{-1}$ & $\left( \zeta_{\Ptop\PZ\Pup}  \right)^2$ \\
		                    & \Ptop\PZ\Pcharm           &   1.684904 $10^{-1}$ & $\left( \zeta_{\Ptop\PZ\Pcharm} \right)^2$ \\
	    \multirow{2}{*}{\eHqt} & \Ptop\PHiggs\Pup       &   1.904399 $10^{-1}$ & $\left( \eta_{\Ptop\PHiggs\Pup}  \right)^2$  \\
		                    & \Ptop\PHiggs\Pcharm       &   1.904065 $10^{-1}$ & $\left( \eta_{\Ptop\PHiggs\Pcharm}  \right)^2$ \\
			\bottomrule
	\end{tabular} 
	\label{tab:partialwidths}
\end{table}
The anomalous single top cross sections are calculated by convolution of the hard scattering matrix elements with the LO order set of \CTEQ 6 partons densities~\cite{Pumplin:2002vw}. The NLO effects are modelled by multiplying each LO cross section by a global $k$-factor. The LO single top production cross section and the global $k$-factors for the top-\PZ production are shown in \tab{tab:STx}. The hard scattering events are then matched to parton showers to \Pythia\ to account for the simulation of the QCD environment relevant for hadronic collisions. 
\begin{table}[htbp]
	\centering
	\caption{Leading order single top production cross section for $\Pproton\Pproton \rightarrow \tZ$ or \tbarZ, where the new physics scale is given in \GeV. The NLO $k-$factors~\cite{Zhang:2011gh} are given in the last column.}
	\begin{tabular}{cllc}
		\toprule
	   Anomalous coupling & \multicolumn{2}{c}{Cross section (\pb)} &  NLO $k-$factor \\ 
		\midrule
	    $\kappa_{\Ptop\Pgluon\Pup} / \Lambda $     &  3.272 $10^7$  & $\left( \kappa_{\Ptop\Pgluon\Pup} / \Lambda \right)^2$ & 1.00 \\
	    $\kappa_{\Ptop\Pgluon\Pcharm} / \Lambda $  &  3.021 $10^6$  & $\left( \kappa_{\Ptop\Pgluon\Pcharm} / \Lambda \right)^2$ & 1.00 \\
	    $\kappa_{\Ptop\Pphoton\Pup} / \Lambda $    &  2.260 $10^5$  & $\left( \kappa_{\Ptop\Pphoton\Pup} / \Lambda \right)^2$ & 1.00 \\
	    $\kappa_{\Ptop\Pphoton\Pcharm} / \Lambda $ &  2.654 $10^4$  & $\left( \kappa_{\Ptop\Pphoton\Pcharm} / \Lambda \right)^2$ & 1.00 \\
	    $\kappa_{\Ptop\PZ\Pup} / \Lambda $         &  1.728 $10^6$  & $\left( \kappa_{\Ptop\PZ\Pup} / \Lambda \right)^2$ & 1.40 \\
	    $\kappa_{\Ptop\PZ\Pcharm} / \Lambda $      &  2.040 $10^5$  & $\left( \kappa_{\Ptop\PZ\Pcharm} / \Lambda \right)^2$ & 1.40 \\
	    $\zeta_{\Ptop\PZ\Pup} $                    &  7.484         & $\left( \zeta_{\Ptop\PZ\Pup} \right)^2$ & 1.40 \\
	    $\zeta_{\Ptop\PZ\Pcharm} $                 &  1.038         & $\left( \zeta_{\Ptop\PZ\Pcharm}  \right)^2$ & 1.40 \\
       \bottomrule
	\end{tabular} 
	\label{tab:STx}
\end{table}

The top pair cross sections are derived from the \SM\ \ttbar\ cross section, calculated with \aMCMG\ at NLO ($\sigma_{\ttbar} = 6.741 \; 10^{2} \pb$), and considering the decay $\ttbar \rightarrow (\Pbottom \PWpm)(\mathrm{X}\Pquark\Ptop)$. The branching ratio $\BR(\Ptop \rightarrow \Pbottom\PWpm)$ is assumed to be equal to one and the FCNC branching ratio is calculated as 
\begin{equation}
 \BR(\Ptop \rightarrow \Pquark\mathrm{X}) = \frac{ \Gamma_{\Ptop \rightarrow \Pquark\mathrm{X}} }{\Gamma_{\Ptop}^{\mathrm{SM}} + \Gamma_{\Ptop}^{\mathrm{FCNC}} }
 		\approx  \frac{ \Gamma_{\Ptop \rightarrow \Pquark\mathrm{X}} }{\Gamma_{\Ptop}^{\mathrm{SM}}} , 
\end{equation}
where $\Gamma_{\Ptop \rightarrow \Pquark\mathrm{X}}$ is given in \tab{tab:partialwidths}, and the assumption $ \Gamma_{\Ptop}^{\mathrm{FCNC}} \ll \Gamma_{\Ptop}^{\mathrm{SM}}$ is made \todo{these partial widths are at LO, how does this relate to NLO that is used? Or is there no difference?}. In \tab{tab:TTx}  the resulting NLO cross sections for the top-Z FCNC interactions are given.  
\begin{table}[htbp]
	\centering
	\caption{ Next to leading order top pair cross section for the top-Z FCNC interactions with with a full leptonic decay. }
	\begin{tabular}{ccll}
		\toprule
		Anomalous coupling & Process &   \multicolumn{2}{c}{Cross section (\pb)}  \\ 
		\midrule
\multirow{2}{*}{$\kappa_{\Ptop\PZ\Pup}/\Lambda$} & $\ttbar \rightarrow (\Pbottom \Pleptonplus\Pneutrino) (\APup \Pleptonplus \Pleptonminus)$ & 2.727008 $10^5$  & $\left( \kappa_{\Ptop\PZ\Pup}/\Lambda \right)^2$ \\
& $\ttbar \rightarrow (\APbottom \Pleptonminus\APneutrino) (\Pup \Pleptonplus \Pleptonminus)$ & 2.727008 $10^5$  & $\left( \kappa_{\Ptop\PZ\Pup}/\Lambda \right)^2$ \\
\multirow{2}{*}{$\kappa_{\Ptop\PZ\Pcharm}/\Lambda$} & $\ttbar \rightarrow (\Pbottom \Pleptonplus\Pneutrino) (\APcharm \Pleptonplus \Pleptonminus)$ &2.726257$10^5$  & $\left( \kappa_{\Ptop\PZ\Pcharm}/\Lambda \right)^2$ \\
 & $\ttbar \rightarrow (\APbottom \Pleptonminus\APneutrino) (\Pcharm \Pleptonplus \Pleptonminus)$ & 2.726257 $10^5$  & $\left( \kappa_{\Ptop\PZ\Pcharm}/\Lambda \right)^2$ \\
\multirow{2}{*}{$\zeta_{\Ptop\PZ\Pup}$} & $\ttbar \rightarrow (\Pbottom \Pleptonplus\Pneutrino) (\APup \Pleptonplus \Pleptonminus)$ & 2.827184   & $\left( \zeta_{\Ptop\PZ\Pup}\right)^2$ \\
 & $\ttbar \rightarrow (\APbottom \Pleptonminus\APneutrino) (\Pup \Pleptonplus \Pleptonminus)$ & 2.827184   & $\left( \zeta_{\Ptop\PZ\Pup}\right)^2$ \\
\multirow{2}{*}{$\zeta_{\Ptop\PZ\Pcharm}$} & $\ttbar \rightarrow (\Pbottom \Pleptonplus\Pneutrino) (\APcharm \Pleptonplus \Pleptonminus)$ & 2.806801  & $\left( \zeta_{\Ptop\PZ\Pcharm}\right)^2$ \\
& $\ttbar \rightarrow (\APbottom \Pleptonminus\APneutrino) (\Pcharm \Pleptonplus \Pleptonminus)$ & 2.806801  & $\left( \zeta_{\Ptop\PZ\Pcharm}\right)^2$ \\
		\bottomrule
	\end{tabular} 
	\label{tab:TTx}
\end{table}



\subsection{Generating \SM\  background events}
The SM \tZq events were generated using the \aMCMG\ generator, interfaced with \Pythia\ version 8.2~\cite{Sjostrand:2014zea}  for parton showering and hadronisation. The \WZ+jets, \ttZ, \tZq, and \ttW\ samples are produced using the \aMCMG (version 5.222)~\cite{Alwall:2014hca}, which includes up to one hadronic jet at next to leading order (NLO) QCD accuracy. Other minor background (e.g. \WW, \ZZ, \tWZ\ and \ttH) are simulated using different generators such as \MG~\cite{Alwall:2011uj},\MS~\cite{Artoisenet:2012st} and \JHU~\cite{Gritsan:2016hjl,Anderson:2013afp,Bolognesi:2012mm,Gao:2010qx}. All events are interfaced to \Pythia\ for parton shower and hadronisation. 

The complete list of \SM\ samples is given in Table \ref{tab:samples} \todocite, along with their cross sections. The cross sections without a reference are coming from the generator with which the sample has been made, for some of them the uncertainties are provided by the Generator Group. For each MC sample, the integrated luminosity that the sample represents is estimated as the number of simulated events divided by the cross section of the generated process. For processes generated with \aMCMG, the effective number of simulated events is used, taking into account positive and negative event weights. The correction factor for those events is defined as
\begin{equation}
\mathrm{C} = \frac{\textnormal{Nb. of pos. weights} + \textnormal{Nb. of neg. weights}}{\textnormal{Nb. of pos. weights} - \textnormal{Nb. of neg. weights}} \times \textnormal{mc baseweight}
\end{equation}

\begin{landscape}
	\begin{table}
		\centering
		\caption{SM MC samples used in this analysis with their corresponding cross section and \aMCMG\ correction C  when applicable. The generators used for each sample are indicated.  }
		\begin{tabular}{llll}
			\toprule
			Process & Generator & Cross section (\pb) & C \\ 
			\midrule
			$\WZ \rightarrow 3\Plepton\Pneutrino$ & \aMCMG +\Pythia & 5.26   & 1.61 \\ 
			
			\tZq\ with $\PZ\rightarrow \Pleptonplus \Pleptonminus$ & \aMCMG +\Pythia & 0.0758  & 3.77 \\ 
			
			\tqH\ with $\PHiggs \rightarrow \ZZ \rightarrow \Pleptonplus \Pleptonminus \Pleptonplus \Pleptonminus$& \JHU+\Pythia&8.80 10$^{-6}$ & - \\ 
			
			\ttW+jets with $\PW\rightarrow \Plepton\Pneutrino$ & \aMCMG +\MS+\Pythia & 0.2043 $\pm$ 0.0020  &1.94 \\ 
			
			
			%/TTWJetsToQQ\_TuneCUETP8M1\_13TeV-amcatnloFXFX-madspin-pythia8/ & 0.4062$\pm$ 0.0021 & -1 \\ 
			 
			$\ttZ\rightarrow 2\Plepton+2\Pneutrino+\mathrm{other}$, with $m_{\Plepton\Plepton}>10 \;\GeV$ & \aMCMG +\Pythia & 0.2529 $\pm$ 0.0004 & 2.15 \\ 
			
			\ttH,no \bbbar\ decays &\Powheg+\Pythia& 0.2151  & - \\ 
		
			\ttH, \bbbar\ decays& \Powheg+\Pythia & 0.2934  & - \\ 
			 
			$\WW\rightarrow 2\Plepton2\Pneutrino$& \Powheg +\Pythia & 12.178  & - \\
			
			$\ZZ\rightarrow 4\Plepton$ & \Powheg+\Pythia & 0.3366 & - \\ 
			 
			\WZZ & \aMCMG +\Pythia&0.05565  & 1.14 \\ 
		
			\ZZZ  & \aMCMG +\Pythia&0.01398  & 1.17 \\ 
		 
			\st\ \tWZ, with $\PZ_{\mu}\rightarrow \Pleptonplus\Pleptonminus$ & \MG +\Pythia&0.001123 & - \\ 
			
			%/ST\_s-channel\_4f\_leptonDecays\_13TeV-amcatnlo-pythia8\_TuneCUETP8M1 & 3 $\times$ 3.36 $^{+0.13}_{-0.12}$  & -1 \\ 
		
			\st\ t-channel \APtop  & \Powheg +\MS +\Pythia& 44.33 $^{+1.76}_{-1.49}$  & - \\ 
		
			\st\ t-channel \Ptop & \Powheg +\MS +\Pythia & 26.38 $^{+1.32}_{-1.18}$   & - \\ 
			
			\st\  $\bar{\mathrm{t}}\PW$ & \Powheg +\Pythia& 35.85 $\pm$ 0.90 (scale) $\pm$ 1.70 (PDF)   & - \\ 
		
			\st\ $\mathrm{t}\PW$ & \Powheg +\Pythia&35.85  $\pm$ 0.90 (scale) $\pm$ 1.70 (PDF) & - \\ 
			
			\ttbar &\Powheg +\Pythia & 831.76 $^{+19.77}_{-29.20}$$^{+35.06}_{-35.06}$   & - \\ 
			
			\DY, with $m_{\Plepton\Plepton}> \;50 \GeV$  & \aMCMG +\Pythia &3 $\times$( 1921.8 $\pm$  0.6 $\pm$ 33.2 ) & 1.49 \\ 
			
			\DY, with $10\; \GeV <m_{\Plepton\Plepton} < 50\; \GeV$ & \MG +\Pythia & 18610  & - \\ 
			\bottomrule 
		\end{tabular} 
		\label{tab:samples}
	\end{table}
\end{landscape}

%\subsection{Parton distribution functions and the hard interaction}
%\subsection{Parton showering}
%\subsection{Hadronisation and decay}
%explanation of jets https://profmattstrassler.com/articles-and-posts/particle-physics-basics/the-known-apparently-elementary-particles/jets-the-manifestation-of-quarks-and-gluons/
%\subsection{Underlying event}
\begin{comment}
%The draft document may be found at this URL: http://cds.cern.ch/record/2261310
%It is version no. 1 entitled:
%`Measurement of the underlying event using inclusive Z boson production in proton-proton collisions at sqrt(s) = 13 TeV`
% http://cms.cern.ch/iCMS/analysisadmin/cadi?ancode=FSQ-16-008
\end{comment}
%\subsection{Event reconstruction and identification}
% ICHEP https://cds.cern.ch/record/2005743
%\section{Event reconstruction}
\section{Multivariate analysis techniques: Boosted Decision Trees}
The need of processing large quantities of data and discriminating between events with largely similar experimental signatures makes multivariate statistical analysis (MVA) a largely used method in the physics community. Multivariate classification methods based on machine learning techniques are a fundamental ingredient to most analyses. The advantage of using a MVA classifier is that it can achieve a better discrimination power with respect to a simple cut and count analysis with a poorly discriminating variables. These variables are referred to as weak variables and have similar distributions for signal and background samples. 
A risk of using MVA classifiers is overtraining.  This happens when there are too many model parameters of an algorithm adjusted to too few data points. This leads to an increase in the classification performance over the objectively achievable one.

There are many software tools that exist for MVA. In this thesis the \texttt{Tool for Multivariate Analysis} (TMVA) \cite{2007physics3039H} is used. This software is an open source project included into \texttt{ROOT}~\cite{Brun:1997pa}. 
%http://idefix.mi.infn.it/~palombo/didattica/AnalisiStatistica/mvaLectures.pdf
All multivariate techniques in TMVA belong to supervised learning algorithms. By training on events for which the outcome is known, a mapping function is determined that describes a classification or an approximation of the underlying behaviour defining the target value (regression). 


In this thesis boosted decision trees (BDT) are employed for the classification of events as implemented in the \texttt{TMVA} framework~\cite{2007physics3039H}. This multivariate techniques is based on a set of decision trees where each yields a binary output depending on the fact that an event is signal- or background-like. The advantage of such a multivariate technique is that several discriminating variables can be combined into a powerful one-dimensional discriminant D. 

In \fig{fig:BDTexample} a schematic view of de decision tree is shown. The starting point is the root node. Then a consecutive set of a total of $i$ questions (nodes) regarding discriminating variables $x_\mathrm{i}$ are asked with only two possible answers per question (binary splits). The decision tree is constructed by training on a dataset for which the outcome is already provided, such as simulation dataset with signal and background processes (supervised learning). For each node a criterion $x_{\mathrm{i}}>C_{\mathrm{i}}$ is found by maximizing the separation gain between nodes 
\begin{equation}
\mathrm{separation}\:\mathrm{gain} \approx \mathrm{gain(parent)} - \mathrm{gain (daughter,Signal))} - \mathrm{gain (daughter,Background))},
\end{equation}
with the gain computed using the Gini index
\begin{equation}
 \mathrm{gain(cell)} \approx p (1-p), 
\end{equation}
where $p$ denotes the purity of a selection $x>C$. This is repeated until the maximum of nodes is reached and at the end of the sequence, the leaf nodes are labelled either signal S or background B, depending on the majority of events that end up on those nodes. 
\begin{figure}[htbp]
	\centering
	\includegraphics[width=0.5\linewidth]{3_Analysis_techniques/Figures/BDT}
	\caption{Schematic view of a decision tree. Figure taken from \cite{2007physics3039H}.}
	\label{fig:BDTexample}
\end{figure}

 Different trees can be combined into a forest where the final output is determined by the majority vote of all trees, forming the sum of so-called weak learners into one strong learner.   From one training collection, trees are derived by reweighting events, and combined into a single classifier as the  weighted average of each individual decision tree. A method for  making such forests is  boosting a tree. In this method, misclassified events are weighted higher so that future learner concentrate on these events. This has as advantage that the response of the decision trees are stabilised against fluctuations in the training sample which enhances the performance. Additionally, the trees can be kept very shallow, in this thesis i = 3, which improves the robustness against overtraining. Examples of such boosting algorithms are Adaptive Boosting (AdaBoost) and Gradient Boosting~\cite{2014arXiv1403.1452M}. In AdaBoost, each weight of the misclassified events are enhanced while reducing the weight of correctly classified events after each training such that  future events learn those better
\begin{equation}
 \alpha_{\mathrm{n+1}} = \left(\frac{1-\epsilon_{\mathrm{n}}}{\epsilon_{\mathrm{n}}}\right)^{\beta}, 
\end{equation}
where $\epsilon_{\mathrm{n}}$ denotes the misclassification error of the current tree n and $\beta$ is a learning rate. The weight $w_{\mathrm{i}}$ at node i is then equal to $w_{\mathrm{i}} = \mathrm{ln}\:\alpha_{\mathrm{i}}$. The final weight is the sum of all classifiers weighted by their errors. The learning rate is typically chosen to be $\beta\leqslant 0.5$ to allow more boosting steps. Gradient boosting has a similar approach and combines a gradient descent with boosting. Instead of fitting the base-learner to the reweighted data as in AdaBoost, it is fitted to the negative gradient vector of the loss function evaluated at the previous node. Misclassified events will result in a majority vote with large gradients of the loss function. Also for the Gradient boost, the learning rate is typically slow, this also known as shrinkage. In this thesis Gradient boost is used with a shrinkage of 0.2-0.3.
%http://www.ccs.neu.edu/home/vip/teach/MLcourse/4_boosting/slides/gradient_boosting.pdf
%https://indico.scc.kit.edu/indico/event/48/session/4/contribution/35/material/slides/0.pdf
%https://arxiv.org/pdf/1403.1452.pdf
%https://www.quora.com/What-is-the-difference-between-gradient-boosting-and-adaboost
% https://people.phys.ethz.ch/~pheno/Lectures2012_StatisticalTools/slides/Chanon2.pdf

In this thesis, the Gradient boost is used in combination with bagging, so-called stochastic gradient boosting. Bagging is a resampling technique draws a subset of events is  from the training data where the same event is allowed to be randomly picked several times from the parent sample. The tree is then trained on this subset and this is repeated many times. It is based on the assumption that sampling from a dataset that follows a distribution is the same as sampling from the distribution itself~\cite{Behnke:2013:DAH:2564838}. If one draws an event out of the parent sample, it is more likely to draw an event out of the phase space that has a high probability density, as the original dataset will have more events in the regions. Since the selected event is kept in the original sample, the parent sample stays unchanged so that randomly extracted samples have the same parent distribution, albeit statistically fluctuated.  Bagging smears over the statistical fluctuations in the training data, making it suitable for stabilising the response of the classifier and increasing the performance by eliminating overtraining.  In stochastic gradient boosting the bagging resampling procedure uses random sub-samples of the training events for growing the trees. 


The discriminating power of a BDT is assessed by analysing the receiver operating statistics (ROC) curve. This curves show the background rejection over the signal efficiency of the remaining sample. By looking at the area under the curve with respect to random guessing (AUC), the best classifier can be identified. This follows the Neyman-Pearson lemma that the best ROC curve is given by the likelihood ratio \like(x|Signal)/\like(x|Background)~\cite{Behnke:2013:DAH:2564838}. No discrimination power will result in an AUC of 0\%, while 50\%  means fully separated event classes. In \fig{fig:ROC} an example of ROC curve is shown. 
\begin{figure}[htbp]
	\centering
	\includegraphics[width=0.5\linewidth]{3_Analysis_techniques/Figures/ROC}
	\caption{Example of ROC curves. In this example, the green method is better than the red one, which is better than the blue one. The dashed line represents a case where there is no separation. Figure taken from \cite{ROC}.}
	\label{fig:roc}
\end{figure}




\section{Template-based fitting}
%\section{Statistics for a high energy particle physicist}
%\label{sec:Stat}

%\subsection{Boosted decision trees}
%\subsection{Confidence levels }

%https://indico.cern.ch/event/614672/timetable/#20170907
%\subsection{Combine limit setting tool}
%\section{Collision event generation}


%
\begin{fmffile}{singletopleptonic}
 \begin{fmfgraph*}(160,40) % width - height
\fmfleft{i1,i2} 
\fmfright{o1,o2,o3,o4,o5}
 \fmf{fermion}{i1,v1,v2,v4,o1}
  \fmf{boson,label=\PZ, label.dist=10}{v2,v6}
  \fmf{fermion}{o5,v6,o4}
   \fmf{gluon}{i2,v1}
   \fmflabel{\Pgluon}{i2}
   \fmflabel{\Pup,\Pcharm}{i1}
   \fmf{boson,label=\PWp,label.dist=-10}{v4,v5}
   \fmffreeze
   \fmf{fermion}{o3,v5,o2}
   	\fmfv{label= ,decor.shape=circle,decor.filled=shaded,decor,size=0.5thick, label.angle=-90}{v2}
   \fmflabel{\Plepton}{o4}
   \fmflabel{\APlepton}{o5}
   \fmflabel{\Pbottom}{o3}
   \fmflabel{\Pneutrino}{o2}
   \fmflabel{\APlepton}{o1}
   \fmf{fermion,label=\Pup,,\Pcharm,label.dist=10}{v1,v2}
   \end{fmfgraph*}
\end{fmffile}


	\begin{fmffile}{toppairglugluleptonic}
	\begin{fmfgraph*}(160,40) % width - height
		\fmfleft{i1,i2} 
		\fmfright{o1,o2,o3,o4,o5,o6}
		\fmflabel{\Pgluon}{i1}
		\fmflabel{\Pgluon}{i2}
		\fmflabel{o1}{o1}
		\fmflabel{o2}{o2}
		\fmflabel{o3}{o3}
		\fmflabel{o4}{o4}
		\fmflabel{o5}{o5}
		\fmflabel{o6}{o6}
		\fmf{gluon}{i1,v1,i2}
		\fmf{boson,tension=-1.5}{v3,v5}
		\fmf{boson,tension=1.5}{v4,v6}
		\fmf{fermion}{o1,v4,v2,v3,o6}
		\fmffreeze
		\fmf{fermion}{o4,v5,o5}
		\fmf{fermion}{o2,v6,o3}
		\fmf{gluon,label=\Pgluon,label.dist=10}{v1,v2}
		\fmfv{label= ,decor.shape=circle,decor.filled=shaded,decor,size=0.2thick, label.angle=-90}{v4}
		% \fmf{fermion,label=\Ptop,label.dist=5}{v2,v3}
		% \fmf{fermion,label=\APtop,label.dist=-20}{v4,v2}
		\fmflabel{}{v3}
		\fmflabel{}{v2}
		\fmflabel{}{v1}
	\end{fmfgraph*}
\end{fmffile}


\chapter{Event reconstruction and selection}
After the detector simulation described in \Sec{sec:eventgeneration}, the simulated data has the exact same format as the real collision data recorded at the CMS experiment. Therefore the same software can be used for the reconstruction of both simulation and real data. In \Sec{sec:reco}, the object reconstruction for physics analysis is shown. After reconstructing the objects, the objects are connected to physics objects need to be identified. This identification is explained in \Sec{sec:id}. A basic event selection is made for selecting signal like events. The necessary event requirement are discussed in \Sec{sec:selection}. 

The analysis uses signal and background regions to constrain the huge \SM\ background compared to the expected signal. \Sec{sec:regions} discusses each region that is entering the analysis. On top of the use of background estimation from control regions, backgrounds that have  prompt leptons  contaminated by real leptons either
from decays of tau leptons or from hadronized mesons or baryons
(collectively commonly referred as ``non-prompt leptons") as well as by
hadrons or jets misidentified as leptons\footnote{These two classes
of contamination will be referred to as not prompt-lepton (\NPL) samples.} are
evaluated with a data-driven method discussed in \Sec{sec:NPL}.

\section{Object Reconstruction}
\label{sec:reco}
In \fig{fig:transversecms}, the particle interaction in a transverse slice of the CMS detector is shown. The particles enter first the tracker where charged particle trajectories, so-called tracks, and origins or vertices are reconstructed from signals (hits) in the sensitive layers. Charged particles get bent by the magnetic field making it able to measure the electric charges and momenta of charged particles. In the ECAL, the electron and photons are absorbed and the corresponding electromagnetic showers are detected as clusters of energy in adjacent  cells.From this, the energy and the direction of the particles can be determined. The charged and neutral hadrons can initiate a hadronic shower in the ECAL that is fully absorbed in the HCAL. The clusters from these showers are also used to estimate the energy and direction. Muons and neutrino's pass through the calorimeters without little to no energy loss. The neutrino's escape the CMS detector undetected while muons produce hits in the muon detectors. 
\begin{figure}
	\centering
	\includegraphics[width=1.\linewidth]{4_EventRecoSelect/Figures/transversecms}
	\caption{Cross-section of the CMS detector with all parts of the detector labelled. This sketch shows the specific particle interactions from a beam interaction reign to the muon detector. The muon and charged pion are positively charged, the electron is negatively charged. Figure taken from~\cite{CMS-PRF-14-001}. }
	\label{fig:transversecms}
\end{figure}

The traditional hadron colliders reconstruction is as follows. The reconstruction of isolated photons and electrons is primarily done by the ECAL, while the identification of muons is based on the muon detectors. Hadrons and photons form jets which are measured by the calorimeters without any contribution from the tracker or muon detectors. Jets can be tagged using the tracker as coming from hadronic \Ptau\ decays or \Pbottom\ hadronisation based on the properties of the properties the relevant charged particle tracks. The missing transverse energy is defined as the vectorial sum of the undetectable particle transverse momenta, and can be reconstructed without any information from the tracker. 
The particle flow (PF)~\cite{CMS-PRF-14-001} reconstruction correlates the tracks and clusters from all detector layers with the identification of each final state particle, and combining the corresponding measurements to reconstruct the properties. Here, the muon is identified by a track in the inner tracker connected to a track in the muon detector as described in \Sec{sec:MuonTrack}. The electrons are identified by a track and ECAL cluster, and not connected to an HCAL cluster as described in \Sec{sec:ElectronTrack}. The ECAL and HCAL clusters without a track link identify the photons and neutral hadrons, while the addition of the tracker determines the energy and direction of a charged hadron. 


Coarse-grained detectors can cause signals of different particles to merge and reduce the ability of identifying and reconstructing the particles. Therefore, particle flow identification requires sufficiently segmented subdetectors such that a global event description is possible. From a list of identified particles that are reconstructed from a combined fit of all relevant measurements, the physics objects are determined. The CMS detector is built to meet to requirements of the particle flow reconstruction. It has an efficient and pure muon identification system, a hermetic HCAL with coarse segmentation, a higher segmented ECAL, a fine-grained tracker and a large magnetic field to separate the calorimeter deposits of charged and neutral particles in jets. 

%http://slideplayer.com/slide/2779564/
%http://slideplayer.com/slide/4496166/
\subsection{Charged particle tracks}
An iterative tracking algorithm is responsible for the reconstruction of the tracks made by charged particles in the inner tracking system. Each iteration consists of four steps~\cite{Bayatian:922757}: the track-seed generation, the pattern recognition algorithm, removal of track-hit ambiguities and a final track fit. 

The seed generation is the first step. It consists of finding reconstructed hits that are usable for seeding the subsequent track-finding algorithm. They are identified from a group of at least three reconstructed hits in the tracker, or from a pair of hits while requiring the origin of the track segment to be compatible with the nominal beam-collision point. Since the pixel has a higher granularity compared to the strip tracker, its seed generation efficiency is higher. The overall efficiency exceeds 99\%.
The second step of each iteration, the pattern recognition algorithm, uses the seeds as a starting point for a Kalman filter method~\cite{FRUHWIRTH1987444,Billoir:1989mh}. This algorithm extrapolates the seed trajectory towards the next tracker layer taking into account the magnetic field and multiple scattering effects. The track parameters are updated when a compatible hit in the next layer is found. This procedure continues until the outermost layer is reached.
Since the Kalman filter method can result in multiple tracks associated to the same seed, or different tracks sharing the same hits, a removal of ambiguities is necessary. This ambiguity resolving is done by removing tracks that are sharing too many hits from the list of track candidates. The tracks with the highest number of hits or with the lowest $\chi^2$ in the track fit is kept. 
The updated track parameters are then refitted using the Kalman filter method, where all hits found in the pattern recognition step are taken into account. The fit is done twice - once outwards from the beam line towards the calorimeters, and inwards from the outermost track hit to the beam line -, improving the estimation of the track parameters. 

All hits that are unambiguously associated to the final track are removed from the list of available hits. In order to associate the remaining hits, the procedure is repeated with looser track reconstruction criteria. The use of the iterative track reconstruction procedure has a high track finding efficiency, where the fake track reconstruction rate is negligible. 
For muons, this results in a global track reconstruction efficiency exceeding 98\%, and 75-98\% for charged hadrons. 

\subsection{Following the Muon's Footsteps}
\label{sec:MuonTrack}
% see http://www.bo.infn.it/sminiato/sm16/03_Mercoledi/Mattina/01_Battilana.pdf
% see https://arxiv.org/pdf/1510.05424.pdf
% see https://twiki.cern.ch/twiki/bin/view/CMSPublic/MuonDPGPublic160729
The muon reconstruction~\cite{Chatrchyan:2012xi} has three subdivisions: local reconstruction, regional reconstruction and global reconstruction. The local reconstruction is performed on individual detector elements such as strip and pixel hits in the inner tracking system, and muon hits and/or segments in the muon chambers. Independent tracks are reconstructed in the inner tracker - called tracker tracks -  and in the muon system, called standalone muon tracks.
Based on these tracks, two reconstructions are considered.

The outside-in approach is referred to as Global Muon reconstruction. 
For each standalone muon track, a inner tracker track is found by comparing the parameters of the two tracks propagated onto a common surface. Combining the hits from the tracker track and the standalone track, gives a fit via the Kalman filter technique~\cite{FRUHWIRTH1987444,Billoir:1989mh} for a global muon track. 

The second approach is an inside-out reconstruction, creating tracker muons. 
All candidate tracker tracks with a \pt$>0.5$ \GeV\ and total momentum p$>2.5$ \GeV\ are extrapolated to the muon system taking into account the magnetic field, the average expected energy losses, and multiple Coulomb scattering in the detector material. The extrapolated track and the muon segments are considered matched when the difference in the position in the x coordinates is smaller than 3~\cm, or when the ratio of this distance to its uncertainty is smaller than four. When at least one muon segment - DT or CSC hits -  matches the extrapolated track, the corresponding tracker track is indicated as a tracker muon. 

For low transverse momenta ($\pt \lesssim$ 5~\GeV), the tracker muon reconstruction is  more efficient than the global muon approach. This is due to the fact that tracker muons only require a single muon  segment in muon system, while the global muon approach requires typically segments in at least two muon stations. These tracker muons are used for identifying muons from the hadronisation of \Pbottom or \Pcharm  quarks. The global muon approach typically improves the tracker reconstruction for $\pt\gtrsim$ 200~\GeV. These are labelled isolated when in a cone of $\Delta R = \sqrt{\Delta\phi^2 + \Delta \eta^2} = 0.3$ around the muon, the sum of the transverse momenta of additional tracker tracks and energy deposits in the calorimeter is less than 10\% of the muon's transverse momentum.
\subsection{The path of the Electron}
\label{sec:ElectronTrack}
% see also https://arxiv.org/pdf/physics/0512097.pdf
% https://cds.cern.ch/record/1563583/files/ATL-PHYS-PROC-2013-206.pdf
% http://cds.cern.ch/record/1704291
The electrons in CMS radiate more than 70\% of their energy in the inner track through bremsstrahlung before reaching the ECAL. This has as consequence that the electron tracks are increasingly curved in the magnetic field as a function of its flight distance. Standard tracking algorithms are based on Kalman filtering which assume that the energy loss is Gaussian distributed, and are therefore not suitable to fit the electron tracks. A different filtering algorithm, the Gaussian sum filter (GSF) \todocite is used in the electron track reconstruction instead. 

In CMS, the electrons are reconstructed in two ways. The older ECAL based tracking is developed to identify high energy, isolated electrons. This tracking algorithm starts from ECAL clusters with a transverse energy above 4~\GeV\ and extrapolates from these cluster the position of the hits in the tracker. In order to account for bremsstrahlung, neighbouring clusters in $\eta$ and $\phi$
are grouped together into a supercluster from which then the direction is determined to find the position of the particles in the tracker. This has as consequence that for electrons or positrons in jets, energy deposits of surrounding particles will be entering the supercluster leading to a wrong position of the electron/positron in the tracker. Another disadvantage of the ECAL based tracking is that for low \pt\ electrons, the trajectories will be very curved and the supercluster will not contain all of the energy deposit, leading to a higher misconstruction rate. 

The faults of the ECAL based tracking are lifted by adding a tracker based algorithm. This algorithm uses all the tracks with a \pt higher than 2~\GeV found with iterative tracking as seeds. Iterative tracking uses the Kalman Filter algorithm several times with an average track reconstruction efficiency but high purity. In contrary with a global combinatorial fit, the iterative tracking accepts tracks with a small transverse momentum that are not leaving any energy in the ECAL, and tracks from particles that only interact with the inner tracker layers. When the electron or positron radiated a small amount of energy, the corresponding track can be reconstructed across the whole tracker and safely propagated to the ECAL surface. When there is a larger amount of enrgy radiated however, the pattern recognition might fail  to accommodate for the change in the electron momentum leading to a track reconstructed with a small number of hits. The solution for this is a preselection based on the $\chi^2$ and number of hits and the selected tracks are fitted again with Gaussian-Sum-Filter which can accommodate substantial enery losses across the trajectory. 

The electron seeds from the ECAL- and tracker-based procedures are merged into a unique collection and are then refitted  by using the summed Gaussian distributions as uncertainty per hit in the track fit. 

The electron efficiency is measured in 8~\TeV\ proton collision data to be better than 93\% for electrons with an ECAL supercluster energy of $E_{\mathrm{T}}>20$~\GeV \todocite. For electrons with an  $E_{\mathrm{T}}>25$~\GeV\  in 13~\TeV\ proton collision data, the effiency is about 96\% \todocite.

%Due to the lack of coverage of the two pixel discs in high \abspsrap range, the efficiency drops. 
%The resolution on the transverse momentum for a 100 \si{ \GeV} charged particle is about 2.0\% (FIX ME). 
% see https://twiki.cern.ch/twiki/bin/view/CMSPublic/TrackingPOGPlots2016
\subsection{Primary Vertex Reconstruction}
\todo{Check text with PFlow paper!}
The primary vertex reconstruction should be able to measure the location of all proton interaction vertices in each event: the signal vertex an all vertices from pile up events. 
It consists of a vertex finding and a vertex fitting algorithm and happens in three steps. Tracks are selected  to be consistent with being produced promptly in the primary interaction by imposing requirements on the track parameters~\cite{Chatrchyan:1704291}. By grouping reconstructed tracks according to the $z$ coordinate of their closest approach to the beam line, vertices for all interaction in the same beam crossing are found, at CMS this is done by a deterministic annealing algorithm~\cite{726788} . On top of this, a vertex fitting algorithm like the Adaptive Vertex fitter~\cite{Waltenberger:1166320}, is performed. This creates the three-dimensional primary-vertex position. With this fit, the contribution from long-lived hadron decays is reduced by down weighting the tracks with a larger distance to the vertex. The primary vertex corresponding to the highest sum of squared track transverse momenta is noted as the point of the main interaction. The resolution on the primary vertex is about 14 \si{ \micro \meter} in $r\phi$ and about 19 \si{ \micro \meter} in the $z$ direction for primary vertices with the sum of the track $p_T > 100$ \si{ \GeV} for 2016 data taking.
% numbers from https://twiki.cern.ch/twiki/bin/view/CMSPublic/TrackingPOGPlotsICHEP2016

\subsection{Calorimeter clusters}
The cluster algorithm in the calorimeter 
\begin{enumerate}
	\item detects and measures the energy and direction of stable neutral particles such as photons and neutral hadron, 
	\item separates neutral particles from charged hadron energy deposits, 
	/item reconstructs and identifies electrons and their bremsstrahlung photons, 
	\item contributes to the energy measurements of charged hadrons that don't have accurate tracks parameters, e.g. for low quality and high transverse momentum tracks. 
\end{enumerate}
The clustering is performed separately in each subdetetector: ECAL barrel and endcaps, HCAL barrel and end caps, and the two preshower layers. The HF has no clustering algorithm since the electromagnetic or hadronic components give rise to an HF EM or HF HAD cluster. 

The clustering algorithm consist of different steps. First seeds are identified when cells have an energy larger than the seeding threshold and lager than their neighbouring cells. Then topological clusters are made by accumulating cells that share at least a corner with a cell already in the cluster and an energy above a cell threshold set to twice the noise level. The third step is a expectation maximization algorithm that reconstructs the cluster~\cite{CMS-PRF-14-001}. This algorithm assumes that  the energy deposits are Gaussian distributed  and is an iterative algorithm with two steps at each iteration. A first step calculated the expected fraction if the energy in a certain step, while the second step performs a maximum likelihood fit. The positions and energies of the Gaussian functions are then taken as cluster parameters. 

The calorimeter clusters are used for reconstructing photons and neutral hadrons. The  clusters that are not in the vicinity of the extrapolated charged tracks are easily identified as neutral hadrons or photons. For the energy deposits that overlap with charged hadrons however, the neutral particle energy deposit can only be detected as an excess over the charged particle deposit. For this reason, a good calibration of the electromagnetic and hadronic calorimeter is  vital. 

The ECAL calibration is performed before the hadron cluster calibration or particle identification\footnote{Specifically electron and photon energy corrections are performed after the identification step.}. For Run 1, the ECAL response to electrons and photons as well as the cell-to-cell relative calibration is determined with test beam data, radio active sources, and cosmic ray measurements. For Run 2, the collision data collected at 7 and 8~\TeV\ was used to refine the calibration. The effect of the thresholds in the clustering algorithm are estimated from simulated single photons with energies varying from 0.25 to 100~\GeV. The photons used for the calibration should not have a conversion prior to their entrance to ensure the calibration of single clusters. In all ECAL regions and for all energies, the calibrated photon energies agree with the true photon energies within 1\%.

In contrary to the photons, the hadrons deposit in general energy in both ECAL and HCAL. Since the calorimeter responce in the HCAL depends on the fraction of shower energy deposited in the ECAL, the ECAL and HCAl cluster energyes are recalibrated together to get an estimate of the true hadron energy. Since now the calibration is done for hadrons, single neutral hadrons such as $K_{\mathrm{L}}^0$ are used for determining the calibration constants. The hadrons interactiong with the tracker material are rejected for the calibration purposes. This calibration is checked with isolated charged hadron selected from early data recorded at $\sqrt{s}=0.9, 2.2$ and 7 \TeV.  

\section{Putting the pieces together}
\label{sec:id}
A link algorithm connects the several PF elements from the various CMS subdetectors. It tests any pair of elements in an event and is restricted to considering nearest neighbours in the $\eta\phi$-plane. The quality of the link is determined via the distance between the two elements and PF blocks of elements are formed from elements with a direct link or indirect link through common elements. 


The link between a central tracker track and a calorimeter clusters is made by extrapolating the tracker track to the two layers of the preshower, the ECAL, and the HCAL. If this extrapolated position is within the cluster area, the two are linked. When there are several ECAL or HCAL clusters for the same track, the link with the smallest distance is kept. A dedicated cluster algorithm accounts for the energy of the photons emitted through bremsstrahlung ar for photons that have converted to an electron-positron pair. \\The ECAL to HCAL cluster and ECAL to preshower cluster links are established when the cluster position in the more granular calorimeter, ECAL or preshower, is in accordance with the cluster envelope of the less granular calorimeter (HCAL or ECAL).  When there are multiple HCAL clusters linked to the same ECAL cluster, the link with the smallest distance is kept. This is also true for multiple ECAL clusters with the same preshower clusters. The ECAL supercluster is linked with the ECAL cluster when they share at least one ECAL cell. \\
Nuclear interactions in the tracker can lead to kinks in hadron trajectories as well as the production of secondary particles. This leads to charged particle tracks linked together via a common displaced vertex. The displaced vertices considered should have at least three tracks, with at most one incoming track, and the invariant mass of the outgoing tracks should exceed 0.2~\GeV. \\
The link between a track and the muon detectors is done via local, regional, and global reconstruction as explained in \Sec{sec:MuonTrack}. 


\section{Particle flow identification}
In each PF block the identification and reconstruction follows a particular order where after each identification and reconstruction the corresponding PF elements (tracks and clusters) are removed from the PF block The muons are the first to be identified and reconstructed. These are reconstructed if their momenta are compatible with corresponding track only momenta. Then the electron and its corresponding brehmstrahung photons, are identified and reconstructed by using of the GSF tracking. At the same time, the energetic and isolated photons are identified as well. The remaining element in the PF block are subjected to a cross identification of charged hadrons, neutral hadrons, and photons that arise from parton fragmentation, hadronisation, and decays in jets. The charged hadron candidate is made from the remaining candidates that have a charged particle track associated with them. Then the charged particle energy fraction is subtracted from the calibrated energy of the linked calorimeter clusters and the remaining energy is assigned to the neutral energy. Depending on the excess of neutral energy in the ECAL and HCAL clusters, a photon or a neutral hadron is assigned respectively. The pseudorapidity range of the inner tracker limits the information on the particles charge to $|\eta| < 2.4$. Outside this range a simplified identification is done for hadronic and electromagnetic candidates only. 

\subsection{Muons}
\label{sec:Muon}
A set of selection requirements based on the global and tracker muon properties is responsible for muon identification. The muons are considered isolated when the additional inner tracks and calorimeter energy deposits within a distance to the muon direction in the $\eta\phi$-plane is smaller than 0.3. The muons coming from charged hadron decays or heavy flavour decays need more stringent criteria. This due to the fact that charged hadrons can be misidentified as muons because of e.g. punch-through, or muons can be seen as charged hadrons, and will absorb the erngy deposits of nearby particles. 
\subsection{Electrons and isolated photons}
\label{sec:Electron}
The electrons and photons are reconstructed together as discussed before. An electron candidate seeded from a GFS track is considered an electron when the linked ECAL cluster is not linked to three or more additional tracks. The photon seeds are ECAL superclusters with transverse energies above 10\GeV\ that have no links with a GSF track. After associating photons from brehmstrahung with the associated electrons, the remaining energy is associated to the photons and the photon direction is taken to be that of the supercluster. The electron direction is chosen to be that of the GSF track and its energy is a combination of the ECAL energy with the momentum of the GSF track. Photons are retained if they are isolated, while electrons should satfisy additional criteria based on a multivariate analysis for isolated and non-isolated electrons. 
\subsection{Hadrons and non-isolated photons}
\label{sec:Hadron}
After muon, electron and isolated photon identification, the remaining particles are hadrons from jet fragmentation and hadronisation. These can show up ad charged hadrons (e.g. $\pi{\pm}$, $\mathrm{K}^{\pm}$, or protons), netutral hadrons (e.g. $\mathrm{K}^{0}_{\mathrm{L}}$ or neutrons), non isolated photons (e.g. from $\pi^0$ decays), and additional muons from early decays of charged hadrons. 

The photons and neutral hadrons are assigned to calorimeter clusters without any link to tracks. When the calorimeter clusters between the ECAL and HCAL are linked, the clusters are assumed to aris from the same hadron shower. if their is not such a link, HCAL clusters are assigned to neutral hadrons, while the ECAL clusters are assigned to photons based on the fact that neutral hadrons leave only 3\% of their energy in the ECAL.Then the HCAL clusters linked with tracks, that are not linked with other HCAL clusters, are assigned to charged hadrons. These tracks can be linked with remaining ECAL clusters. 

Hadron interactions can result in the creation of extra particles originating from a secondary vertex. These extra particles are identified by having a common secondary vertex and replaced in the PF list as one single primary charged hadron. 


\subsection{Post processing}
\label{sec:Postprocess}
After identification and reconstruction of all particles as described above. An artificial large missing transverse momentum \ptmisvec\ can be reconstructed. The cause of the \ptmisvec\ is mostly misidentified or misreconstructed high-\pt\ muons originating from cosmic rays, misconstruction of the muon's momentum, or punch-throuvh charged hadrons. A post processing step is applied to solve this \ptmisvec. Events with genuine large \ptmisvec\ due to the presence of neutrino's are unaffected by this post processing.
\section{Physics object reconstruction and identification}
\label{sec:PhysicsObject}
The particle flow objects are used for building physics objects that are used for analysis. These objects are jets, muons, electrons, photons, taus and missing transverse momentum \ptmisvec. They are used to compute other quantities such as particle isolation and have extra requirements that are analysis dependent. In the following section, only the physics objects used throughout this thesis are discussed. 

\subsection{Muons}
\label{sec:MuonID}
The muon candidates used for analysis in this thesis correspond to the tight working point. This working point yields the most genuine muons and rejects falsely reconstructed ones. Detail reports on the performance can be found in \todocite.

In order to reject objects wrongly reconstructed as muons from hadron showers that reach he muon system (punch-throughs), the global muon fit is required to include at least one valid hit in the muon chambers and for which at least two muon segments in two muon stations is present. Additionally, the muon tracks should have a global fit yielding a goodness-of-fit of $\chi^2 / \mathrm{ndof} < 10$. The decay of muons in flight is suppressed by requiring at least one pixel hit in the muon track. Furthermore, a minimum of five hits in the tracker is required. Cosmic muons and muons originating from pile up interactions are rejected by constricting the distance of the muon with respect to the primary vertex by putting limits on $d_{\mathrm{x,y}}< 2$ \mm\ and $d_{\mathrm{z}}<5$ \mm. In \fig{fig:tightid}, the muon efficiencies for data and simulation is presented. These efficiencies are estimated from tag-and-probe methods that select $\PZ \rightarrow \Pmuon \APmuon$ and tag one muon that passes the identification criteria. The other muon is used as probe and one measures how many times it passes the identification criteria to get the efficiency. Overall, the efficiency is about 95-100\%, with two drops due to the crack between the wheels of the DT system. The differences between data and simulation are corrected by applying \pt- and $\eta$-dependent scale factors ($\epsilon_{\mathrm{data}/\epsilon_{\mathrm{MC}}}$) to simulated events. 
\begin{figure}[htbp]
	\centering
	\includegraphics[width=0.495\linewidth]{4_EventRecoSelect/Figures/TightIDvseta}
	\includegraphics[width=0.495\linewidth]{4_EventRecoSelect/Figures/LooseIDvseta}
	\caption{Comparison of the muon tight ID (left) and loose ID (right) efficiencies in data and simulation as a function of the pseudorapidity of the muon using the full 2016 dataset. Figure taken from \cite{CMS-DP-2017-007}.}
	\label{fig:tightid}
\end{figure}

In addition to the identification criteria, the muons are required to be spatially isolated from EM and hadronic activity.  The lepton isolation is defined as estimating the total transverse energy of the particles emitted around the  direction of the lepton by defining a cone of radius $\Delta R$ in $\eta\phi$ plane around the lepton direction. Then a summed energy is calculated from the charged hadrons (CH), neutral hadrons (NH), photons (\Pphoton), excluding the lepton itself. This sum is then corrected to remove the enrgy coming from pile up interactions. The relative isolation for muons \iso\ is defined as \todocite
\begin{equation}
 \iso_{\Pmu} = \frac{\sum \pt(CH) + \mathrm{max}\left(0., \sum \Et(NH), \sum \Et(\Pphoton) - 0.5 \times \sum \Et (CH)\right)}{\pt(\Pmu)},
\end{equation}
where a cone of $\Delta R = 0.4$ is adopted and the pile up mitigation is based on the  \dbeta\  correction. The \dbeta correction estimates the pile up energy as half of the contribution coming from charged hadrons. For tight ID muons, this relative isolation should $\iso_{\Pmu} \leq 0.15$, while for loose muons this should be $\iso_{\Pmu} \leq0.25$ % The chosen value for β is motivated by assuming equal production rates for the (π+,π0,π−) isospin triplet leading to a ratio of 1/2 for the production of neutral pions over charged ones. 
In \fig{fig:muoniso}, the isolation efficiencies as a function of the pseudo rapidties using the tag and probe method are shown. The efficiencies are 85-100\% with a decline for low-\pt\ muons since they are most likely coming from hadronic or heavy flavour decays. The differences between data and simulation are accounted for by apply $\eta$- and \pt-dependent scale factors on the simulation. 
\begin{figure}
	\centering
	\includegraphics[width=0.494\linewidth]{4_EventRecoSelect/Figures/TightIsovsot}
	\includegraphics[width=0.494\linewidth]{4_EventRecoSelect/Figures/TightIsovseta}
	\caption{Comparison of the muon tight isolation requirement with the muon tight ID  efficiencies in data and simulation as a function of the transvers emomentum (left) or  pseudorapidity (right) of the muon using the full 2016 dataset. Figure taken from \cite{CMS-DP-2017-007}.}
	\label{fig:muoniso}
\end{figure}

\subsection{Jets}
Jets are reconstructed using the \antikt\ algorithm \todocite. Based on the clustering used by this algorithm, the jets are denoted as
\begin{enumerate}
 \item PF jets containing all particles reconstructed by the PF algorithm, 
 \item  Calo jets made from the sum of the ECAL and HCAL energy deposits in the calorimeter towers, 
 \item Ref jets made from all stable particles produced by the event generator with the exclusion of neutrinos. 
\end{enumerate}

\subsection{Jets with displaced vertices }

\subsection{Electrons}
\section{Summary of corrections}
\begin{enumerate}
	\item Muon ID, Iso see \Sec{sec:MuonID}
\end{enumerate}
\begin{comment}
% Jet energy scale and resolution in the CMS experiment in pp collisions at 8 TeV
% http://iopscience.iop.org/article/10.1088/1748-0221/12/02/P02014/meta
% atlas http://inspirehep.net/record/1519834

% photobn http://iopscience.iop.org/article/10.1088/1748-0221/10/08/P08010/pdf
\subsection{The particle flow event reconstruction method}
% https://cds.cern.ch/record/2237475?ln=en
% atlas http://inspirehep.net/record/1520722
\subsection{Identification of particles}
\subsubsection{Muon reco and ID}
% trigger and good explenation of ID https://arxiv.org/pdf/1206.4071.pdf
% https://cds.cern.ch/record/2257968/files/DP2017_007.pdf
\subsubsection{Electron reco and ID}
% https://cds.cern.ch/record/2255497/files/DP2017_004.pdf
% https://cds.cern.ch/record/2255497?ln=en
\subsubsection{Jet reco and ID of b quarks}
% jet algorithms 
% http://cms.cern.ch/iCMS/analysisadmin/cadi?ancode=JME-16-003

% Identification of b and c jets in the CMS experiment at the LHC Run 2
% http://cms.cern.ch/iCMS/analysisadmin/cadilines?line=BTV-16-002
% SF https://twiki.cern.ch/twiki/bin/view/CMS/BtagRecommendation80XReReco
%Identification of b quark jets at the CMS Experiment in the LHC Run 2
% https://cds.cern.ch/record/2138504?ln=en

%Identification of c-quark jets at the CMS experiment
%https://cds.cern.ch/record/2205149?ln=en

\subsubsection{Missing transverse energy reconstruction}
\subsection{Calibrations and corrections}
%CMS has been taking collision data since the 13TeV startup of the LHC on 3 June. During this period, the CMS magnet has been kept off due to an issue with the cooling system, so the beams have been used to calibrate and time-in the electronics of the various parts of the detector. These operations, which are largely independent of the magnetic field, are now complete. Meanwhile, the data collected with zero magnetic field can be used for fundamental research, like the measurement of the multiplicity of charged particles produced at the new collision energy of 13 TeV. The issue with the magnet cooling system was identified in the final preparatory phase leading to collisions in the LHC. While preparing for beam in CMS, a problem was found in the system that feeds liquid helium to the CMS superconducting magnet. The problem was diagnosed to be due to oil, which is used in the initial compression stages, reaching the so-called 'cold-box’ of the cryogenic system. The cold-box is a complex system with several sets of filters protecting three turbines along the path of the helium towards the magnet. In order to clean the oil contamination essentially all components of the cold-box have been extracted and replaced. Analysis confirms that there is no oil contamination in the CMS magnet itself or risk to its operation during 2015. The cold-box of is now being stabilised after the cleaning intervention and is being brought back to operational conditions. CMS is confident that, following the LHC technical stop and the beam conditioning run that will start at the end of this week, after the low-intensity and commissioning period, the full magnetic field will be available for the 13 TeV LHC run.
\end{comment}






\section{Event selection}
\label{sec:selection}
\section{Regions and channels}
\label{sec:regions}
\section{Data driven background simulation}
\label{sec:NPL}

\chapter{The search for FCNC involving a top quark and a Z boson}
\section{Construction of template distributions}
\section{Systematic uncertainties}
\section{Limit setting procedure}
\section{Result and discussion}
\begin{comment}
\section{Model assumptions}
% based on Aguilar
% implemented in Feynrules
\section{Data and simulation}
\subsection{Standard Model Background simulation}

% ZZ 4 l cross sec http://cms.cern.ch/iCMS/analysisadmin/cadi?ancode=SMP-16-017
% ttH the PAS of HIG-17-004
%http://cms.cern.ch/iCMS/analysisadmin/cadi?ancode=HIG-17-004

% DY Drell-Yan mass differential cross section measurement for electrons and muons at 13 TeV
% http://cms.cern.ch/iCMS/analysisadmin/cadi?ancode=SMP-17-001


%The physics analysis
%`tHq multilepton with 2016 dataset`
%(CADI entry HIG-17-005)
%will be presented for approval at the HIG PAG meeting on Tue, May 2, 2017.
%The corresponding documentation can be found on CADI at:
%http://cms.cern.ch/iCMS/analysisadmin/cadi?ancode=HIG-17-005

The physics analysis
`Measurement of the top pair-production in association with a W or Z boson in pp collisions at 13 TeV`
(CADI entry TOP-17-005)
will be presented for approval at the physics meeting on Thu, May 4, 2017:

https://indico.cern.ch/event/635522/#12-top-17-005-measurement-of-t

The corresponding documentation can be found on CADI at:
http://cms.cern.ch/iCMS/analysisadmin/cadi?ancode=TOP-17-005

% http://utils.paranoiaworks.org/diacriticsremover/

\subsection{FCNC signal simulation}
In this thesis, two scenarios are being studied: one being the top-up interactions and the second one being top-charm interactions. For a given flavour of light quark \Pquark, all left-handed chiral parameters were set to zero and all right-handed set to one
\subsection{Trigger requirements}
\section{Baseline event selection}
\section{Data driven background estimation}
\section{Regions and channels}
\section{Construction of template distributions}
\section{Systematic uncertainties}
% theoretical uncertainties at lhc 
% https://cds.cern.ch/record/888430/files/note05_013.pdf

%https://twiki.cern.ch/twiki/bin/view/CMS/TopSystematics#Parton_shower_uncertainties
\section{Limit setting procedure}
\section{Result and discussion}
\end{comment}
\chapter{Conclusion and outlook}
\begin{figure}
	\centering
	\includegraphics[width=0.7\linewidth]{Future}
	\caption{}
	\label{fig:future}
\end{figure}



% -------------------------------------------------- %
%    appendix and stuff                              %
% -------------------------------------------------- %

\backmatter

%\bibliography{Bibliography/Chapter2_references}  % use with BibTeX
\hypersetup{urlcolor=darkgreen}
\printbibliography  % use with BibLaTeX
\hypersetup{urlcolor=darkblue}
%\printglossaries
%\printglossary[type=\acronymtype, toctitle=\myacronymtitle, title=\myacronymtitle]  % print only one glossary, and define its title (\myacronymtitle is defined in _settings.tex)
%\printglossaries  % print all glossaries available

\end{document}
