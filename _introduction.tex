\chapter{Introduction}

The Standard Model (SM) of particle physics is the theory of fundamental particles and their interactions. This theory has been experimentally confirmed and all its predicted particles have been found. Despite its many successes, the Standard Model has its shortcomings and can not explain phenomena such as neutrino masses, dark matter, dark energy or gravity. The heaviest particle in the Standard Model is the top quark  and physicists believe that it has an enhanced sensitivity to various new particles and interactions suggested by beyond the Standard Model theories. Its lifetime is so short that it doesn't form bound states, making it possible to study the bare quark. Furthermore, the top quark  has a distinct signature since it almost exclusively decays to a \PW\ boson and a bottom quark. This makes the top quark the ideal candidate to directly study quark properties. On top of this, many beyond the Standard Model physics phenomena are investigated by measuring the production rate of top quarks by studying the $\PW\Ptop\Pbottom$ vertex and interactions that are heavily suppressed in the Standard Model are researched. The Large Hadron Collider (LHC) is a top quark factory, producing a large number of events containing top quarks. At the proton collision points, experiments are placed to study these collisions. The work presented in the thesis uses the data collected by one of such an experiment, the Compact Muon Solenoid (CMS), and investigates flavour changing neutral currents (FCNC). 


In the 1960s, the charged current process occurring when a charged kaon decays to a muon ($ \mathrm{K}^+ \rightarrow \APmuon \Pnum$) was a well known process. The neutral current process $ \mathrm{K}^{0}_L \rightarrow \APmuon \Pmuon$ was however not observed. This suppression of neutral currents with respected to charged currents was baffling the physicists of that time. At that time, only three different quarks were known and although the existence of a fourth quark was proposed, there was no evidence for it yet. The GIM mechanism~\cite{PhysRevD.2.1285,Maiani:2013fpa}, proposed in 1970, was the theory providing a satisfying explanation for the suppression of neutral current processes compared to charged current processes, via a fourth quark with specific couplings to the other quarks. This brought a revolution in physics, and  the GIM mechanism combined with then already existing measurements provided indirect indications of the charm quark before it was directly observed in $J/\Psi$ meson decays~\cite{Aubert:1974js}. This confirmed that flavour changing neutral currents, that change the flavour of a fermion without altering its electric charge, are highly suppressed in the Standard Model. 

The first evidence for flavour changing neutral currents was provided in 2005 by the CDF experiment~\cite{Acosta:2005eu}, by looking at $B^0_s\rightarrow \phi\phi$ decays. However, it was the large production rate of \Pbottom\ hadrons at the LHC that made it possible for the LHCb and CMS collaboration to measure the FCNC decays of \Pbottom\ hadrons~\cite{CMS:2014xfa}. Their combined data  provided the first discovery of FCNC decays with the $B_s^0 \rightarrow\Pmuon\APmuon$ decay. This observation agreed with the Standard Model prediction and has put stringent limits on many beyond the Standard Model theories. Recent interpretations for bottom to strange quark transitions have even found strong hints for new physics~\cite{Capdevila:2017bsm}. These interpretations are based on the discrepancy from the Standard Model prediction, measured in 2003 by the LHCb collaboration for $B \rightarrow K^* \Pmuon \APmuon$~\cite{Descotes-Genon:2013vna,Aaij:2013qta,Descotes-Genon:2013wba,DescotesGenon:2012zf}, and confirmed in 2015~\cite{Aaij:2015oid}. Additionally, a deficit for the branching ratios of several decays such as  $B_{\mathrm{S}} \rightarrow \phi\Pmuon \APmuon$  with respect to the Standard Model predictions have been found in 2013~\cite{Aaij:2013aln} and 2015~\cite{Aaij:2015esa} by the LHCb collaboration. The Belle experiment confirmed these measurements in 2016~\cite{Abdesselam:2016llu,Wehle:2016yoi}. Another remarkably observation made by the LHCb collaboration in 2014~\cite{Aaij:2014ora}, where was observed that the deviations from the Standard Mode were stronger for the $B \rightarrow K^* \Pmuon \APmuon$ decays than for $B \rightarrow K^* \Pelectron \APelectron$. This hints for a violation of lepton flavour universality. The Belle experiment has confirmed these hints by measuring the lepton flavour universality violating terms in 2016~\cite{Wehle:2016yoi} and also the update with more data of LHCb in 2016~\cite{Descotes-Genon:2015uva,Aaij:2016flj} still points towards this direction. The CMS and ATLAS collaborations have also measured $B^0_s\rightarrow \phi\phi$ decays~\cite{ATLAS-CONF-2017-023,Sirunyan:2287571}, confirming the LHCb results. In Ref.~\cite{Capdevila:2017bsm}, several beyond the Standard Model theory have been identified as possible candidates for a new all-comprehending theory and time will point out which ones will prevail. %https://arxiv.org/pdf/1704.05340.pdf

%https://www.nature.com/nature/journal/v546/n7657/pdf/nature21721.pdf
%Many extensions of the Standard Model induce sizeable flavour changing neutral current interactions, making it a sensitive probe for physics beyond the Standard Model.
%By performing precise measurements of the known decays of hadrons, one can study the processes that occur via the weak force. These are particularly interesting since such decays are possible via virtual transient particles and heavy new particles can  cause large deviations from the Standard Model predications for the decay rates and have an influence the dynamics of the decay products. An effective Lagrangian can be used to describe the effects of such an interaction when the responsible new physics is heavier than the top, making it possible to search for top flavour violating interactions in a model independent way. 
%The first limits on the flavour violating top-quark production is set by the LEP experiments by studying the process $\Pep\Pem\rightarrow \Ptop\Pquark$, which is sensitive to the $\Ptop\Pgamma\Pup$ and $\Ptop\PZ\Pup$ couplings \todocite. Also at HERA, the same vertices were studied. The hadron colliders are able to investigate all top FCNC couplings and the anomalous single top quark production as well as the top quark pair production with an anomalous decay have been performed both at Tevatron \todocite, as well as the LHC. 



The search presented in this thesis is performed on data collected by the Compact Muon Solenoid experiment at a centre-of-mass energy of 13 \TeV, resulting in 35.9 \fbinv\ of integrated luminosity. The anomalous couplings of a \PZ\ boson to a top quark and an up or charm quark are being investigated for three-lepton signatures in the final state of proton collisions. In the Standard Model, the branching fraction for a top quark decaying into a charm or up quark and a \PZ\ boson is of the order of $10^{-14}$~\cite{AguilarSaavedra:2004wm,PhysRevD.2.1285}. Several extensions of the Standard Model however enhance the \FCNC\ branching fractions and can be probed at the LHC \cite{AguilarSaavedra:2004wm}. Additional to the potential signals in top quark pair events, these interactions may result in flavour changing single top quark production through a tZq vertex.  Up to this moment,  the searches performed on top quark related flavour changing neutral currents are done at the Tevatron by the CDF~\cite{PhysRevLett.101.192002} and D0~\cite{Abazov:2010qk} collaborations, as well as at the LHC by the ATLAS~\cite{Aad:2015uza,Aad:2015gea} and CMS~\cite{Sirunyan:2017kkr,Chatrchyan:2013nwa,Khachatryan:2015att,Sirunyan:2017kkr}  collaborations. Their limits are still far from the sensitivity required to reach the Standard Model predictions,  leaving a large phase space to reveal or exclude new physics phenomena. 


The first chapter of this thesis introduces the Standard Model of particle physics as well as a brief introduction into effective field theories. The end of this chapter focusses on flavour changing neutral currents involving top-Z-quark interactions and overviews the current experimental limits.  The second chapter gives a description of the Large Hadron Collider at CERN, as well as the CMS experiment. Its coordinate system, hardware, and data acquisition are discussed. The analysis techniques used for the search  presented in this thesis are discussed in the third chapter. Here, the simulation of proton collisions is explained, and an introduction is given to Boosted Decision Trees as well as the statistical methodology used for the search. In the fourth chapter, the reconstruction and identification of particles within CMS is explained. At the end, an overview is given of the corrections that one needs to apply for making the simulation agree with data. The event selection and categorisation used for the search are explained in the fifth chapter. Here the analysis strategy is set and the statistical independent datasets are defined. In chapter six, the templates of variable distributions used for the limits on the FCNC \tZq\ coupling are discussed and the results are presented. The final chapter concludes the search and an outlook to the future is given. 