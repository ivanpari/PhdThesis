The Standard Model of particle physics is a theory of fundamental particles and their interactions. This theory has been experimentally confirmed and all particles predicted by this theory have been found. Nonetheless, this theory has its shortcomings and can not explain phenomena such as neutrino masses or gravity. The heaviest particle of the Standard Model is the top quark and physicists believe that it has an enhanced sensitivity to various new particles and interactions suggested by new physics theories. The top quark has a very short lifetime, hence it does not form bound states and it can be used to study the bare quark. Furthermore, it has a distinct signature since it almost exclusively decays to a W boson and a bottom quark. This makes the top quark an ideal candidate for the study of quark properties. Also, many beyond the Standard Model physics phenomena are studied by measuring the production rate of top quarks for probing the Wtb vertex, and interactions that are heavily suppressed in the Standard Model can be investigated. The Large Hadron Collider (LHC) is a proton collider, producing a large number of events containing top quarks. At the proton collision points, experiments are placed to study the collisions. The search presented in this thesis is performed on data collected by the Compact Muon Solenoid (CMS) experiment at a centre-of-mass energy of 13 TeV, resulting in 35.9 fb$^{-1}$ of integrated luminosity. 


Flavour changing neutral currents (FCNC) are highly suppressed in the Standard Model. However, many beyond the Standard Model theories enhance their probability. In this thesis, a search in three lepton final states is performed for the production of single top quarks via the tZq vertex, with q=c or u, for the top quark pair processes where one of the top quarks decays through this vertex.  No significant deviation with respect to the predicted background is observed and upper limits at 95\% confidence level are obtained. The observed (expected) upper limits at 95$\%$ confidence level  on the branching fractions are: ${\mathcal{B}}(t \rightarrow uZ) < 0.024\%$ ($0.015\%$) and ${\mathcal{B}}(t \rightarrow cZ) < 0.045\%$ (0.037$\%$), assuming one non-vanishing FCNC coupling at a time. 


Significant improvements are developed with respect to previous searches, namely by using other kinematic variables as input into the BDT as well as a better handle on the non prompt lepton background.  The sensitivity of this search exceeds that of previous analysis at the CMS experiment, and is comparable with the sensitivity obtained at the ATLAS experiment. Although the limits on the FCNC interactions with a tZq vertex start to probe physics beyond the Standard Model, the branching fractions predicted within the Standard Model remain out of reach. 