

The Standard Model of particle physics is a theory of fundamental particles and their interactions. This theory has been experimentally confirmed and all particles predicted by this theory have been found. Nonetheless, this theory has it shortcomings and can not explain phenomena such as neutrino masses or gravity. The heaviest particle of the Standard Model is the top quark and physicists believe that it has an enhanced sensitivity to various new particles and interactions suggested by beyond the Standard Model theories. The top quark has a very short lifetime, so short that it doesn't form bound states and it can be used to study the bare quark. Furthermore, it has distinct signature since it almost exclusively decays to a \PW\ boson and a bottom quark. This makes the top quark an ideal candidate for the study of quark properties. Also, many beyond the Standard Model physics phenomena are researched by measuring the production rate of top quarks for studying the $\PW\Ptop\Pbottom$ vertex, and interactions that are heavily suppressed in the Standard Model can be investigated. The Large Hadron Collider is a top quark factory, producing a large number of events containing top quarks. At the proton collision points, experiments are placed to study the collisions. The search presented in this thesis is performed on data collected by the Compact Muon Solenoid experiment at a centre-of-mass energy of 13 \TeV, resulting in 35.9 \fbinv\ of integrated luminosity. 


Flavour changing neutral currents are highly suppressed in the Standard Model. However, many beyond the Standard Model theories enhance their probability. In this thesis, a search in three lepton final states is performed for the production of single top quarks via the \tZq\ vertex, with $\Pquark=\Pcharm, \Pup$, or the top quark pair processes where one of the top quarks decay through this vertex.  No significant deviation with respect to the predicted background is observed and upper limits at 95\% confidence level are placed. The observed (expected) upper limits at 95$\%$ confidence level  on the branching fractions are: ${\mathcal{B}}(t \rightarrow uZ) < 0.024\%$ ($0.015\%$) and ${\mathcal{B}}(t \rightarrow cZ) < 0.045\%$ (0.037$\%$), assuming one non-vanishing coupling at a time. 

Significant improvements are developed with respect to previous searches, namely by using other kinematic variables as input into the BDT as well as a better handle on the not prompt lepton background.  The expected limit at 95\% CL for the FCNC \Zut\ interaction is is more stringent than the expected limit (0.024\%) of the current best observed limit (0.017\%) set at a centre-of-mass energy of 13~\TeV\ by ATLAS.  The  observed (expected) limit on the \Zct\ interaction set by ATLAS is 0.023\% (0.032\%) and its expected limit is comparable with the expected limit presented in this search.  For the FCNC interactions with a \tZq\ vertex, the branching fractions predicted within the Standard Model or beyond the Standard Model theories are still out of reach. 

