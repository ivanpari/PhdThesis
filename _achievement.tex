\chapter*{Achievements and contributions}
During the course of my doctoral studies, I have been part of the tZ(q) analysis group at the CMS collaboration and have contributed to the paper of the \tZq\ SM and FCNC search at a centre-of-mass energy of 8 TeV by CMS: 
\vspace{+2ex}
\chapterprecishere{Search for associated production of a Z boson with a single top quark and for tZ flavour-
	changing interactions in pp collisions at $ \sqrt{s}=8 $ TeV, by CMS Collaboration (Albert
	M Sirunyan et al.), arXiv:1702.01404 [hep-ex], JHEP 1707 (2017) 003.}
\vspace{-2ex}
Proceeding on this collaboration, I have become one of the convenors of the tZ(q) \SM\ and FCNC reseach group. I coordinated the weekly meetings and the collaborations between the several institutes and universities: the Centro de Investigaciones Energ\'eticas, Medioambientales y Tecnol\'ogicas (CIEMAT), Interuniversity Institute of High Energies (IIHE), the Institut Pluridisciplinaire Hubert Curien (IPHC), the National Centre for Physics (NCP) and the Brunel University London. Amongst the research topics in this group are the \SM\ tZq as well as the \FCNC\ \tZq\ search in the three-lepton and the dilepton final states. Both three-lepton searches have been published, where the \SM\ search is published as
\vspace{+2ex}
\chapterprecishere{Evidence for the standard model production of a Z boson with a single top quark in pp collisions at $\sqrt{s}=13$ \TeV, by CMS Collaboration (Albert
	M Sirunyan et al.), CMS-PAS-TOP-16-020, CDS record 2284830,}
\vspace{-2ex}
and the FCNC search is published as
\vspace{+2ex}
\chapterprecishere{Search for flavour changing neutral currents in top quark production and decays with three-lepton final state using the data collected at sqrt(s) = 13 TeV, by CMS Collaboration (Albert
	M Sirunyan et al.), CMS-PAS-TOP-17-017, CDS record 2292045.}
\vspace{-2ex}
The latter comprises the research presented in this thesis and is also where my main contribution lies. A complete list of my 292 peer reviewed publications can be found below\footnote{\url{http://inspirehep.net/search?ln=en&ln=en&p=find+i+van+parijs&of=hb&action_search=Search&sf=earliestdate&so=d&rm=&rg=25&sc=0}}.

\newpage\thispagestyle{empty}
As part of my research I attended many conferences and workshops. I was one of the speakers at  the General Scientific Meeting of the Belgian Physical Society in 2016, where I presented
\vspace{+2ex}
\chapterprecishere{Flavor changing neutral currents of top quarks at the LHC, a probe for new physics, by Isis Van Parijs}
\vspace{-2ex}
Furthermore, I represented the tZ(q) research group at the 3rd Single Top Workshop in 2016 and gave the talk
\vspace{+2ex}
\chapterprecishere{tZq in tri-lepton SM+FCNC, by Isis Van Parijs for the analysis group.}
\vspace{-2ex}
Additionally, I was chosen by the CMS collaboration to represent the CMS collaboration as one of the speakers at the 9th International Workshop on Top Quark Physics in 2016. Here I presented 
\vspace{+2ex}
\chapterprecishere{Multilepton signatures from tZq interactions in the SM and top-FCNC,  by CMS Collaboration (Isis Van Parijs
	for the collaboration)}
\vspace{-2ex}
for which conference proceedings are published at 
\vspace{+2ex}
\chapterprecishere{Three lepton signatures from tZq interactions in the SM and top-
	FCNC at the CMS experiment at $\sqrt{s}$ = 8 TeV, by CMS Collaboration (Isis Van Parijs
	for the collaboration), arXiv:1703.00162 [hep-ex].}
\vspace{-2ex}


At the CMS collaboration, it is common to take on experimental physics responsibility (EPR). I had the opportunity to be part of the Tracker team, where my work was mainly focussed on the Silicon Strip tracker. I have done shifts of detector expert on call, where I was the responsible for the Silicon Strip tracker on site and coordinated the daily activities. During LS1, the tracker has been made ready to operate at lower temperatures. The CMS cooling plant was refurbished, new methods for vapour sealing and insulation were applied, and high-precision sensors to monitor the humidity and temperature were placed. I had the opportunity to be part of the placement of the vapour sealing and sensors. Furthermore, I was involved in hardware testing such as pressure testing of the new dry gas system. Additionally, I have been involved in the calibration of the detector control units (DCUs). The CMS tracker silicon micro-strip sensors are exposed to high levels of radiation which causes an increases in the detector leakage current as well as a change in the detector depletion voltage. The DCUs measures the temperatures, voltages and leakage currents and makes it able to control the radiation damage. A good calibration is therefore necessary to spot any abnormalities.  I was also involved in the Phase 2 upgrades of the tracker. For the HL-LHC, the tracker will be included in the level-1 trigger in order to handle the increase in luminosity. Therefore, the tracker geometry is altered and the maps of the tracker used for Detector Quality Monitoring (DQM) had to be updated. Lastly, I was DQM expert on call, where I represented the first line of expertise for the daily data quality monitoring during Run 2. 

As member of the IIHE institute, I was involved in the bachelor student excursion to CERN and did recruiting of new physics students at the open days of the university as well as at high schools. Furthermore, I was part of the scientific jury of 90$\degree$ South. This is an initiative  where middle school and high school students could participate in a scientific competition for the opportunity to take part in a physics experiment at the South pole. Additionally, for my work at CMS, I have taken the first aid as well as the fire extinguisher courses to provide a safer working environment. 
