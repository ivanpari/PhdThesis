 A basic event selection is made for selecting signal like events. The necessary event requirement are discussed in \Sec{sec:selection}. 

The analysis uses signal and background regions to constrain the huge \SM\ background compared to the expected signal. \Sec{sec:regions} discusses each region that is entering the analysis. On top of the use of background estimation from control regions, backgrounds that have  prompt leptons  contaminated by real leptons either
from decays of tau leptons or from hadronized mesons or baryons
(collectively commonly referred as ``non-prompt leptons") as well as by
hadrons or jets misidentified as leptons\footnote{These two classes
	of contamination will be referred to as not prompt-lepton (\NPL) samples.} are
evaluated with a data-driven method discussed in \Sec{sec:NPL}.
\section{Baseline event selection and filters}
Trigger and filters
\section{Event selection}
\label{sec:selection}
%met filter %http://cds.cern.ch/record/2205284/files/JME-16-004-pas.pdf
\section{Effect of the corrections in dilepton events}
\section{Regions and channels}
\label{sec:regions}
\section{Data driven background simulation}
\label{sec:NPL}