After the detector simulation described in \Sec{sec:eventgeneration}, the simulated data has the exact same format as the real collision data recorded at the CMS experiment. Therefore the same software can be used for the reconstruction of both simulation and real data. In \Sec{sec:reco}, the event reconstruction for physics analysis is shown. After reconstructing events, a basic event selection is made for selecting signal like events. The necesarry event requirement are discussed in \Sec{sec:selection}. 

The analysis uses signal and background regions to constrain the huge \SM\ background compared to the expected signal. \Sec{sec:regions} discusses each region that is entering the analysis. On top of the use of background estimation from control regions, backgrounds that have  prompt leptons  contaminated by real leptons either
from decays of tau leptons or from hadronized mesons or baryons
(collectively commonly referred as ``non-prompt leptons") as well as by
hadrons or jets misidentified as leptons\footnote{These two classes
of contamination will be referred to as not prompt-lepton (\NPL) samples.} are
evaluated with a data-driven method discussed in \Sec{sec:NPL}.

\section{Event reconstruction}
\label{sec:reco}




\begin{comment}
% Jet energy scale and resolution in the CMS experiment in pp collisions at 8 TeV
% http://iopscience.iop.org/article/10.1088/1748-0221/12/02/P02014/meta
% atlas http://inspirehep.net/record/1519834

% photobn http://iopscience.iop.org/article/10.1088/1748-0221/10/08/P08010/pdf
\subsection{The particle flow event reconstruction method}
% https://cds.cern.ch/record/2237475?ln=en
% atlas http://inspirehep.net/record/1520722
\subsection{Identification of particles}
\subsubsection{Muon reco and ID}
% trigger and good explenation of ID https://arxiv.org/pdf/1206.4071.pdf
% https://cds.cern.ch/record/2257968/files/DP2017_007.pdf
\subsubsection{Electron reco and ID}
% https://cds.cern.ch/record/2255497/files/DP2017_004.pdf
% https://cds.cern.ch/record/2255497?ln=en
\subsubsection{Jet reco and ID of b quarks}
% jet algorithms 
% http://cms.cern.ch/iCMS/analysisadmin/cadi?ancode=JME-16-003

% Identification of b and c jets in the CMS experiment at the LHC Run 2
% http://cms.cern.ch/iCMS/analysisadmin/cadilines?line=BTV-16-002
% SF https://twiki.cern.ch/twiki/bin/view/CMS/BtagRecommendation80XReReco
%Identification of b quark jets at the CMS Experiment in the LHC Run 2
% https://cds.cern.ch/record/2138504?ln=en

%Identification of c-quark jets at the CMS experiment
%https://cds.cern.ch/record/2205149?ln=en

\subsubsection{Missing transverse energy reconstruction}
\subsection{Calibrations and corrections}
%CMS has been taking collision data since the 13TeV startup of the LHC on 3 June. During this period, the CMS magnet has been kept off due to an issue with the cooling system, so the beams have been used to calibrate and time-in the electronics of the various parts of the detector. These operations, which are largely independent of the magnetic field, are now complete. Meanwhile, the data collected with zero magnetic field can be used for fundamental research, like the measurement of the multiplicity of charged particles produced at the new collision energy of 13 TeV. The issue with the magnet cooling system was identified in the final preparatory phase leading to collisions in the LHC. While preparing for beam in CMS, a problem was found in the system that feeds liquid helium to the CMS superconducting magnet. The problem was diagnosed to be due to oil, which is used in the initial compression stages, reaching the so-called 'cold-box’ of the cryogenic system. The cold-box is a complex system with several sets of filters protecting three turbines along the path of the helium towards the magnet. In order to clean the oil contamination essentially all components of the cold-box have been extracted and replaced. Analysis confirms that there is no oil contamination in the CMS magnet itself or risk to its operation during 2015. The cold-box of is now being stabilised after the cleaning intervention and is being brought back to operational conditions. CMS is confident that, following the LHC technical stop and the beam conditioning run that will start at the end of this week, after the low-intensity and commissioning period, the full magnetic field will be available for the 13 TeV LHC run.
\end{comment}




\section{Event selection}
\label{sec:selection}
\section{Regions and channels}
\label{sec:regions}
\section{Data driven background simulation}
\label{sec:NPL}