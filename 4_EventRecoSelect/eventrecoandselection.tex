After the detector simulation described in \Sec{sec:eventgeneration}, the simulated data has the exact same format as the real collision data recorded at the CMS experiment. Therefore the same software can be used for the reconstruction of both simulation and real data. In \Sec{sec:reco}, the event reconstruction for physics analysis is shown. After reconstructing events, a basic event selection is made for selecting signal like events. The necesarry event requirement are discussed in \Sec{sec:selection}. 

The analysis uses signal and background regions to constrain the huge \SM\ background compared to the expected signal. \Sec{sec:regions} discusses each region that is entering the analysis. On top of the use of background estimation from control regions, backgrounds that have  prompt leptons  contaminated by real leptons either
from decays of tau leptons or from hadronized mesons or baryons
(collectively commonly referred as ``non-prompt leptons") as well as by
hadrons or jets misidentified as leptons\footnote{These two classes
of contamination will be referred to as not prompt-lepton (\NPL) samples.} are
evaluated with a data-driven method discussed in \Sec{sec:NPL}.

\section{Event reconstruction}
\label{sec:reco}

%http://slideplayer.com/slide/2779564/
%http://slideplayer.com/slide/4496166/
\subsubsection*{Muon reconstruction}
% see http://www.bo.infn.it/sminiato/sm16/03_Mercoledi/Mattina/01_Battilana.pdf
% see https://arxiv.org/pdf/1510.05424.pdf
% see https://twiki.cern.ch/twiki/bin/view/CMSPublic/MuonDPGPublic160729
The muon reconstruction~\cite{Chatrchyan:2012xi} has three subdivisions: local reconstruction, regional reconstruction and global reconstruction. 
The local reconstruction is performed on individual detector elements such as strip and pixel hits in the inner tracking system, and muon hits and/or segments in the muon chambers. Independent tracks are reconstructed in the inner tracker - called tracker tracks -  and in the muon system, called standalone tracks.
Based on these tracks, two reconstructions are considered.
The outside-in approach is referred to as Global Muon reconstruction. 
For each standalone track, a tracker track is found by comparing the parameters of the two tracks propagated onto a common surface. Combining the hits from the tracker track and the standalone track, gives a fit via the Kalman filter technique~\cite{FRUHWIRTH1987444,Billoir:1989mh} for a global muon track. 
The second approach is an inside-out reconstruction, creating tracker muons. 
All candidate tracker tracks are extrapolated to the muon system taking into account the magnetic field, the average expected energy losses, and multiple Coulomb scattering in the detector material. When at least one muon segment - DT or CSC hits -  matches the extrapolated track, the corresponding tracker track is indicated as a tracker muon. 

For low transverse momenta ($p_T \lesssim$ 5 \si{ \GeV}), the tracker muon reconstruction is  more efficient than the global muon approach. This is due to the fact that tracker muons only require a single muon  segment in muon system, while the global muon approach requires typically segments in at least two muon stations. The global muon approach typically improves the tracker reconstruction for $p_T\gtrsim$ 200 \si{ \GeV}.
\subsubsection*{Track reconstruction}
% see also https://arxiv.org/pdf/physics/0512097.pdf
% https://cds.cern.ch/record/1563583/files/ATL-PHYS-PROC-2013-206.pdf
% http://cds.cern.ch/record/1704291

An iterative tracking algorithm is responsible for the reconstruction of the tracks made by charged particles in the inner tracking system. Each iteration consists of four steps\cite{Bayatian:922757}: the track-seed generation, the pattern recognition algorithm, removal of track-hit ambiguities and a final track fit. 

The seed generation is the first step. It consists of finding reconstructed hits that are usable for seeding the subsequent track-finding algorithm. They are identified from a group of at least three reconstructed hits in the tracker, or from a pair of hits while requiring the origin of the track segment to be compatible with the nominal beam-collision point. Since the pixel has a higher granularity compared to the strip tracker, its seed generation efficiency is higher. The overall efficiency exceeds 99\%.
The second step of each iteration, the pattern recognition algorithm, uses the seeds as a starting point for a Kalman filter method~\cite{FRUHWIRTH1987444,Billoir:1989mh}. This algorithm extrapolates the seed trajectory towards the next tracker layer taking into account the magnetic field and multiple scattering effects. The track parameters are updated when a compatible hit in the next layer is found. This procedure continues until the outermost layer us reached.
Since the Kalman filter method can result in multiple tracks associated to the same seed, or different tracks sharing the same hits, a removal of ambiguities is necessary. This ambiguity resolving is done by removing tracks that are sharing too many hits from the list of track candidates. The tracks with highest number of hits or with the lowest $\chi^2$ if the track fit is kept. 
The updated track parameters are then refitted using the Kalman filter method, where all hits found in the pattern recognition step are taken into account. The fit is done twice - once outwards from the beam line towards the calorimeters, and inwards from the outermost track hit to the beam line -, improving the estimation of the track parameters. 

All hits that are unambiguously associated to the final track are removed from the list of available hits. In order to associate the remaining hits, the procedure is repeated with looser track reconstruction criteria. The use of the iterative track reconstruction procedure has a high track finding efficiency, where the fake track reconstruction rate is negligible. 
For muons, this results in a global track reconstruction efficiency exceeding 98\%, and 75-98\% for charged hadrons. 
%Due to the lack of coverage of the two pixel discs in high \abspsrap range, the efficiency drops. 
%The resolution on the transverse momentum for a 100 \si{ \GeV} charged particle is about 2.0\% (FIX ME). 
% see https://twiki.cern.ch/twiki/bin/view/CMSPublic/TrackingPOGPlots2016
\subsubsection*{Primary vertex reconstruction}
The primary vertex reconstruction should be able to meausre the location of all proton interaction vertices in each event: the signal vertex an all vertices from pile up events. 
It consists of a vertex finding and a vertex fitting algorithm and happens in three steps. Tracks are selected  to be consistent with being produced promptly in the primary interaction by imposing requirements on the track parameters\cite{Chatrchyan:1704291} By grouping reconstructed tracks according to the $z$ coordinate of their closest approach to the beam line, vertices for all interaction in the same beam crossing are found, at CMS this is done by a deterministic annealing algorithm~\cite{726788} . On top of this, a vertex fitting algorithm like the Adaptive Vertex fitter~\cite{Waltenberger:1166320}, is performed. This creates the three-dimensional primary-vertex position. With this fit, the contribution from long-lived hadron decays is reduced by down weighting the tracks with a larger distance to the vertex. The primary vertex corresponding to the highest sum of squared track transverse momenta is noted as the point of the main interaction. The resolution on the primary vertex is about 14 \si{ \micro \meter} in $r\phi$ and about 19 \si{ \micro \meter} in the $z$ direction for primary vertices with the sum of the track $p_T > 100$ \si{ \GeV} for 2016 data taking.
% numbers from https://twiki.cern.ch/twiki/bin/view/CMSPublic/TrackingPOGPlotsICHEP2016



\begin{comment}
% Jet energy scale and resolution in the CMS experiment in pp collisions at 8 TeV
% http://iopscience.iop.org/article/10.1088/1748-0221/12/02/P02014/meta
% atlas http://inspirehep.net/record/1519834

% photobn http://iopscience.iop.org/article/10.1088/1748-0221/10/08/P08010/pdf
\subsection{The particle flow event reconstruction method}
% https://cds.cern.ch/record/2237475?ln=en
% atlas http://inspirehep.net/record/1520722
\subsection{Identification of particles}
\subsubsection{Muon reco and ID}
% trigger and good explenation of ID https://arxiv.org/pdf/1206.4071.pdf
% https://cds.cern.ch/record/2257968/files/DP2017_007.pdf
\subsubsection{Electron reco and ID}
% https://cds.cern.ch/record/2255497/files/DP2017_004.pdf
% https://cds.cern.ch/record/2255497?ln=en
\subsubsection{Jet reco and ID of b quarks}
% jet algorithms 
% http://cms.cern.ch/iCMS/analysisadmin/cadi?ancode=JME-16-003

% Identification of b and c jets in the CMS experiment at the LHC Run 2
% http://cms.cern.ch/iCMS/analysisadmin/cadilines?line=BTV-16-002
% SF https://twiki.cern.ch/twiki/bin/view/CMS/BtagRecommendation80XReReco
%Identification of b quark jets at the CMS Experiment in the LHC Run 2
% https://cds.cern.ch/record/2138504?ln=en

%Identification of c-quark jets at the CMS experiment
%https://cds.cern.ch/record/2205149?ln=en

\subsubsection{Missing transverse energy reconstruction}
\subsection{Calibrations and corrections}
%CMS has been taking collision data since the 13TeV startup of the LHC on 3 June. During this period, the CMS magnet has been kept off due to an issue with the cooling system, so the beams have been used to calibrate and time-in the electronics of the various parts of the detector. These operations, which are largely independent of the magnetic field, are now complete. Meanwhile, the data collected with zero magnetic field can be used for fundamental research, like the measurement of the multiplicity of charged particles produced at the new collision energy of 13 TeV. The issue with the magnet cooling system was identified in the final preparatory phase leading to collisions in the LHC. While preparing for beam in CMS, a problem was found in the system that feeds liquid helium to the CMS superconducting magnet. The problem was diagnosed to be due to oil, which is used in the initial compression stages, reaching the so-called 'cold-box’ of the cryogenic system. The cold-box is a complex system with several sets of filters protecting three turbines along the path of the helium towards the magnet. In order to clean the oil contamination essentially all components of the cold-box have been extracted and replaced. Analysis confirms that there is no oil contamination in the CMS magnet itself or risk to its operation during 2015. The cold-box of is now being stabilised after the cleaning intervention and is being brought back to operational conditions. CMS is confident that, following the LHC technical stop and the beam conditioning run that will start at the end of this week, after the low-intensity and commissioning period, the full magnetic field will be available for the 13 TeV LHC run.
\end{comment}




\section{Event selection}
\label{sec:selection}
\section{Regions and channels}
\label{sec:regions}
\section{Data driven background simulation}
\label{sec:NPL}