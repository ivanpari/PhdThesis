\chapterprecishere{\Titledutch}
Het Standaard Model van de deeltjesfysica is een theorie over fundamentele deeltjes en hun interacties. Deze theorie werd door diverse experimenten geconfirmeerd en al de deeltjes die deze theorie voorspelt zijn ook daadwerkelijk ontdekt. Ondanks zijn vele successen heeft deze theorie echter ook zijn tekortkomingen. Fenomenen zoals neutrino massa's of zwaartekracht blijven onverklaard. De top quark is het zwaarste deeltje in het Standaard Model en doet fysici geloven dat het een verhoogde gevoeligheid heeft voor nieuwe deeltjes en interacties voorspeld door nieuwe theorie\"en. De top quark heeft zo een korte levensduur, dat het onstnapt aan de vorming van gebonden toestanden en het mogelijk is om rechtstreeks de top quark eigenschappen te onderzoeken. 
Veel nieuwe fysica theorie\"en worden onderzocht  via de studie van de productie van top quarks en nemen zo het Wtb interactiepunt onder de loep. De  ``Large Hadron Collider'' (LHC) is een proton versneller en produceert een zeer groot aantal proton botsingen die resulteren in top quarks.  Dit maakt de LHC een ideale plek om top quark fenomenen te onderzoeken. Op elk proton 
botsingspunt zijn experimenten geplaatst om de botsingen te onderzoeken.  Het onderzoek gepresenteerd in deze thesis omvat de data verzameld door het ``Compact Muon Solenoid'' (CMS) experiment aan een massamiddelpuntsenergie van  13 TeV, resulterend in een ge\"integreerde luminositeit van 35.9 fb$^{-1}$. 


Smaakveranderende neutrale stromen zijn erg beperkt in het Standaard Model, tot op de hoogte dat deze niet waarneembaar zijn. Vele nieuwe fysica theorie\"en verhogen echter hun waarschijnlijkheid en maken de observatie mogelijk. Het onderzoek gepresenteerd in de thesis kijkt naar interacties resulterend in een eindtoestand met drie leptonen veroorzaakt door de productie van enkelvoudige top quarks door middel van een tZq interactiepunt, of door het verval via een tZq interactiepunt van een top quark in een top quark paar gebeurtenis. De additionele quark kan ofwel een charm quark, ofwel een up quark zijn. Er is geen significante afwijking gevonden ten op zichte van de achtergronden en bovenlimieten met 95\% betrouwbaarheid zijn bepaald. De geobserveerde (verwachte) bovenlimieten met een 95$\%$ betrouwbaarheid op de vertakkingsfractie zijn: ${\mathcal{B}}(t \rightarrow uZ) < 2.4\times 10^{-4}$ ($1.5\times 10^{-4}$) en ${\mathcal{B}}(t \rightarrow cZ) < 4.5\times 10^{-4}$ (3.7$\times 10^{-4}$), waarbij verondersteld wordt dat er enkel \'e\'en niet nulle smaakveranderende koppeling aanwezig is. \newpage
\thispagestyle{empty}
De vooruitgang ten opzichte van vroeger onderzoek komt door het gebruik van andere kinematische variabelen alsook  een beter begrip van de achtergrond gevormd door niet prompte leptonen.  Dit onderzoek heeft een  hogere gevoeligheid ten op zichte van eerder onderzoek aan het CMS experiment en is vergelijkbaar met de gevoeligheid verkregen met het ATLAS experiment. Ondanks dat 
de limieten op de smaakveranderende interacties via een tZq vertex nieuwe fysica beginnen te naderen, blijven de vertakkingsfracties voorspeld door het Standaard Model buiten bereik. 