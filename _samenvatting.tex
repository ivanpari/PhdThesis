\chapterprecishere{\Titledutch}
Het Standaard Model van de deeltjes fysica is een theorie handelend over fundamentele deeltjes en hun interacties. Tot op heden is dit de beste theorie om de natuur om ons heen te beschrijven. Het is experimenteel geconfirmeerd en al de deeltjes die deze theorie voorspelt zijn ook daadwerkelijk gevonden. Ondanks zijn vele successen heeft deze theorie echter ook zijn tekortkomingen. De fenomenen zoals neutrino massa's of zwaartekracht blijven onverklaard. \\
De top quark is het zwaarste deeltje in het Standaard Model en doet fysici geloven dat het een verhoogde gevoeligheid heeft voor nieuwe deeltjes en interacties voorspeld door theori\"en van nieuwe fysica. De top quark heeft zo\'n korte levensduur, dat het onstnapt van de formatie van gebonden toestanden en het mogelijk is om rechtstreeks de top quark eigenschappen te onderzoeken. 
Veel nieuwe fysica theori\"en worden onderzocht door naar de productie snelheid van top quarks te kijken en zo het $\PW\Ptop\Pbottom$ interactiepunt onder de loep te nemen. Verder kunnen ook interacties die beperkt worden in het Standaard Model zo worden geanalyseerd. \\
De  ``Large Hadron Collider '' (LHC) is een zogenaamde top quark `` fabriek '' en produceerd een zeer groot aantal interacties die resulteren in top quarks.  Dit maakt de LHC een ideale plek om top quark fenomenen te onderzoeken en op elk proton collisie punt zijn experimenten geplaatst om de collisies te onderzoeken.  Het onderzoek gepresenteerd in deze thesis omvat de data verzameld door het ``Compact Muon Solenoid '' (CMS) experiment aan een massamiddelpuntsenergie van  13 \TeV, resulerend in een ge\"integreerde luminositeit van 35.9 \fbinv. 


Smaakveranderende neutral stromen zijn erg beperkt in het Standaard Model, tot op de mate dat deze niet waarneembaar zijn. Desondanks, vele nieuwe fysica theori\"en verhogen hun waarschijnlijkheid en maken de observatie mogelijk. Het onderzoek gepresenteerd in de thesis kijkt naar interacties resulterend in eindtoestand met drie leptonen veroorzaakt door de productie van enkelvoudige top quarks door middel van een \tZq\ interactiepunt, of door het verval via een \tZq\ interactiepunt van een top quark komende van een top quark paar productie. De additionele quark kan ofwel een charm quark, ofwel een up quark zijn. Er is geen significante afwijking ten op zicht van de achtergronden gevonden en bovenlimieten met 95\% betrouwbaarheisinterval zijn geplaatst. De geobserveerde (verwachtte) bovenlimieten met een 95$\%$ betrouwbaarheidsinterval op de vertakkingsfractie zijn: ${\mathcal{B}}(t \rightarrow uZ) < 0.024\%$ ($0.015\%$) en ${\mathcal{B}}(t \rightarrow cZ) < 0.045\%$ (0.037$\%$), waarbij verondersteld wordt dat er enkel \'e\'en niet nulle koppeling aanwezig is. 

De vooruitgang ten opzichte van vroegere onderzoeken komt door het gebruik van andere kinematische variabelen alsook  een betere grip op de \NPL\ achtergrond.  De verwachtte limiet op de  FCNC \Zut\ interactie is sterker dan deze van de huidige sterkste waargenomen (verwachtte) limiet  van 0.017\% (0.024\%) aan een massamiddelpuntsenergie van 13 \TeV\ door  ATLAS~\cite{ATLAS-CONF-2017-070}.  De waargenomen (verwachtte) limiet voor de \Zct\ interactie verkregen door ATLAS is 0.023\% (0.032\%) en is vergelijkbaar met de verwachtte limiet van het onderzoek in deze thesis. Voor  FCNC interacties,  waar een  \tZq\ interactie plaatsvindt, blijven de vertakkingsfractie voorspeld door het Standaard Model en nieuwe fysica theori\"en nog buiten het bereik van ontdekking of uitsluiting. 